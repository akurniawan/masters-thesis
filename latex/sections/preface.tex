\chapter*{Introduction}
\addcontentsline{toc}{chapter}{Introduction}
Pre-trained language models \citewithpar{devlin2018bert,howard2018universal} have received considerable attention in recent years. These models are trained on large-scale corpora and then fine-tuned for a particular downstream task. This method allows pre-trained models to perform well across various natural language processing tasks. One of the most successful models is BERT \citewithpar{devlin2018bert}. BERT has been most extensively used for common natural language understanding (NLU) tasks. It has been shown that BERT can achieve great performance with relatively straightforward fine-tuning, especially for \hl{classification-like tasks} \citewithpar{devlin2018bert,wang2018glue,Rajpurkar2016SQuAD1Q}.

For natural language generation (NLG), incorporating BERT is still challenging. According to \normcite{zhu2020incorporating}, simply incorporating BERT into the encoder side of the seq2seq architecture can hurt the performance. On the decoder side, BERT does not quite fit either because the bidirectional nature of the model was significantly different from the objective of the conditional language model (predicting the next word).

Fine-tuning all BERT's parameters is inefficient, given that there are approximately 200 million parameters in a single model of BERT. Naive fine-tuning also often results in catastrophic forgetting, where the models forget the previous knowledge they have acquired while improving on the new domain \citewithpar{mccloskey1989catastrophic,yogatama2019learning}. This may explain why it is considered harmful to simply fine-tune an initialized encoder component with BERT. It is also known that fine-tuning large pre-trained language models could result in unstable and fragile performance on small datasets.

Adapter is an alternative approach that allows for fine-tuning a model without altering the original network weights \citewithpar{houlsby2019parameter,bapna2019simple}. By leveraging adapters, one can reduce the number of parameters updated in fine-tuning and make the process computationally less expensive while achieving similar results. Another useful property of the approach with adapters is that they are more robust against catastrophic forgetting than fine-tuning \citewithpar{han2021robust}.

This work uses BERT and its variants as the base pre-trained models and fine-tune them with adapters. We evaluate the models on machine translation with the following objectives:
\begin{itemize}
    \item We conduct a study to understand the contribution of good representation in the pre-trained language model when fine-tuning using adapters.
    \item We conduct a study to evaluate the effectiveness of adapters in the seq2seq framework by putting them only in the encoder or the decoder.
    \item We experiment with down-scaling the pre-trained model size and try to recover the performance from being comparable to the full-sized model.
\end{itemize}

\section*{Thesis Organization}

\paragraph{Chapter 1} discusses the theoretical background of machine translation, transfer learning, and a brief overview of the current state of using adapters in various setups.

\paragraph{Chapter 2} reviews the previous related work on transfer learning from models that were pre-trained on language model objectives and the usage of adapters in various disciplines within text and speech domains.

\paragraph{Chapter 3} describes the dataset that we use to train language models and machine translation. We then explain the pre-processing of the dataset and the tokenization to construct the vocabularies. Finally, we describe the framework for the experiments and the automatic evaluation metric.

\paragraph{Chapter 4} presents our attempt to use adapters in machine translation setup. This chapter focuses on the contribution of pre-trained representations when fine-tuning with adapters. We discuss the result of our experiments by referring to the automatic evaluation metric as well as providing our own manual analysis. Furthermore, we discuss the limitation of incorporating BERT in machine translation by providing the translation output errors.

\paragraph{Chapter 5} presents our attempt to understand the effectiveness of adapters and the impact of the pre-trained weights for adapters by placing them only in the encoder or decoder. We then continue the experiments by down-scaling BERT to half of the size and trying to recover the adapter's performance so that it is comparable to the full-sized model. We provide discussions of the phenomenon that happens when reducing the size of the original BERT model.

\paragraph{Conclusion} summarizes our findings from the experiments.