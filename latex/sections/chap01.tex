\chapter{Background and Motivation}

In this chapter, we summarize the background and motivation underlying this work. In Section \ref{sec:bm_smt}, we briefly review the approaches to machine translation (MT) and the comparison between them.

Section \ref{sec:bm_nmt} reviews the formulation of neural machine translation as well as the frameworks utilized to solve the problem.

% Section \ref{sec:weight_init} defines weight initialization in deep neural network and the importance of finding the proper initialization to be able to gain optimum performance.

Section \ref{sec:bm_tl} discuss transfer learning (TL) as well as its variants.

Finally, Section \ref{sec:bm_adapters} discusses the adapter module as an alternative methodology for fine-tuning the model and provides reasons why we focus our research on using adapters.

\section{Machine Translation}
\label{sec:bm_smt}
On a fundamental level, MT performs the substitution of words from one language to another. However, it is challenging to produce a good translation based on the substitution alone, as understanding the whole sentence that includes phrases and surrounding words in the target language is needed. The problem is exacerbated as words may have more than one meaning, and it is sometimes challenging to determine the exact relation between expressions across languages.

The three most commonly used approaches in MT are rule-based, statistical (SMT) and neural (NMT). Due to the significant effort in manually collecting good dictionary and grammatical rules, demands on a more automatic approach such as SMT or NMT seems more appealing.
Before NMT, a variant of SMT, namely phrase-based machine translation (PBMT), had been the state-of-the-art for German-to-English language pairs. \cite{bojar2015proceeding} also shows that PBMT has a good performance in different language pairs. \cite{blunsom2013recurrent} introduced the first end-to-end neural network for machine translation with an encoder-decoder structure. Their approach encodes a given source text into a continuous vector space and further transforms the  vector into the target language.

While NMT has been the primary technique used in various MT challenges, such as WMT, there are some advantages and disadvantages. According to \cite{koehn2017nmt}, NMT suffers from the following phenomena:
\begin{itemize}
    \item In an out-of-domain scenario, NMT systems suffer more from quality loss than PBMT. The authors found that the model tends to sacrifice adequacy over fluency completely.

    \item NMT requires a large amount of data. It is problematic when low-resource languages are involved in the evaluation.

    \item NMT still shows weakness in the translation quality for low-frequency words despite its better performance. A similar problem also occurs in PBMT, where the model's performance suffers when low-frequency words occur in the sentence. The problem is especially apparent when the word is entirely unknown.

    \item Difficulties in translating long sentences. NMT can perform well in short sentences up to 60 words. However, longer sentences show a lower quality of translation.

    \item Model interpretability. As opposed to PBMT, it is not easy to interpret the behaviour of NMT due to the complexity introduced in the hyperparameter and the model architecture. Furthermore, the training of NMT is also non-deterministic due to random parameter initialization.
\end{itemize}

Despite its shortcomings, NMT also shows a promising direction in MT. \cite{machavcek2018enriching} mentioned that the difference is apparent in the output fluency. They mentioned that PBMT models suffer from double negation and translation in morphologically rich languages. These problems cause little to no problem at all in NMT models.

\section{Neural Machine Translation}
\label{sec:bm_nmt}
We define the translation problem as a mapping function $t$ of sentences from a given source language $S$ and target language $T$ from parallel corpus, where $t : S \rightarrow T$. A parallel corpus is a pair of sentences in two different languages where one sentence in $T$ corresponds to its equivalent in language $S$. The goal of function $t$ is to find the highest probability of word $y \in T$ from $x \in S$, where $t(x) = argmax_y(p(y|x))$. The probability $p(y|x)$ is the probability estimation given by the NMT model.

The following sections show a quick overview of recent NMT models. They include architecture, advantages, and drawbacks.

\subsection{RNN Seq2Seq}
\cite{sutskever2014sequence} proposes seq2seq model that employs Recurrent Neural Network (RNN) as the fundamental architecture. The architecture is a straightforward extension of language model problem. Essentially, the model sequentially predicts the next word given all previous words while also incorporating features generated from the source language.

In MT, the approach is modified by using two similar model architectures for language $S$ and $T$. For language $S$ we call this component \texttt{encoder} and for $T$ we call it \texttt{decoder}.
The task of the \texttt{encoder} is to produce a vector representation of the input sentence from source language $x \in S$. We define the input sentence as a sequence of tokens from a fixed set of vocabulary $x \in S$. The tokens are then transformed into a vector representation by an embedding matrix. These representations consumed by an RNN resulted in a new representation that combines features from the embedding and its context.
\texttt{decoder} has a similar functionality as the \texttt{encoder} where it uses a sequence of tokens from $y \in T$ as its inputs. The \texttt{decoder} incorporates additional features from the the last vector of the \texttt{encoder} and uses it throughout the entire sequence in target language. For further illustration, we refer to Image \ref{img:rnnseq2seq}.

\begin{figure}[h]
    {\includegraphics[width=0.95\textwidth]{img/rnnseq2seq.png}}
    \centering
    \caption{Illustration of seq2seq architecture\protect\footnotemark[1].}
    \label{img:rnnseq2seq}
\end{figure}

\cite{cho2014properties} found several disadvantages of using this variant of seq2seq framework as they found that the models' performance decreased when the length of the source sentence increased.
From the previous paragraph, we recall that the \texttt{decoder} only uses the last vector of \texttt{encoder} as the additional feature.
The vector is a fixed-size vector whose size is defined prior to the training. Hence, the vector could be less informative when the source sentence grows due to the missing information that it has to sacrifice during the encoding. In essence, this framework forces the \textt{encoder} to create a good representation by combining the features from all the words in the source sentence within a single vector with a limited size.

\footnotetext[1]{Figure reprinted from \protect\url{https://www.guru99.com/seq2seq-model.html}.}

Furthermore, RNNs also suffer on another problem where the gradient can be extremely small or large. These problems are often mentioned as vanishing and exploding gradient problems. When the model's gradient is extremely small, RNNs can not learn from the data effectively, especially in the long-range dependencies setup. On the other hand, when the gradient is extremely large, it can affect the weight parameters by moving them far away from the optimal space. This would disrupt the learning process and cause the model to fail to learn. This problem can happen in seq2seq architecture as we backpropagate the weights from the end of the \texttt{decoder} to the beginning of the \texttt{encoder}.

\subsection{Seq2Seq with Attention}
An extension of the encoder-decoder model is proposed by jointly learning to translate and align words in source language to the target language\cite{bahdanau2015nmt}. This method learns to identify the most relevant sources of information in a source sentence and proceeds to use the context vectors associated with the source positions to predict a target word.

\begin{figure}[h]
    {\includegraphics[width=0.95\textwidth]{img/attseq2seq.png}}
    \centering
    \caption{Illustration of seq2seq architecture with attention\protect\footnotemark[2].}
    \label{img:attseq2seq}
\end{figure}

This method eliminates the need for a neural model to learn to squash the whole sentence into a fixed-length vector. Instead of trying to encode a whole sentence as the method proposed by \cite{sutskever2014sequence}, it selects a subset of vectors from the source sentence that is deemed to contain the most relevant information to be used while decoding the translated message. This proves to allow the model to provide better predictions in long sentences. We can see the illustration of this architecture from Figure \ref{img:attseq2seq}.

\footnotetext[2]{Figure reprinted from \protect\url{https://lena-voita.github.io/nlp_course/seq2seq_and_attention.html}.}

The proposed approach achieves significantly better translation performance than the basic encoder-decoder model. The improvement is especially evident in longer sentences. The model shows comparable performance to or close to the phrase-based system in the En-Fr pair.

The motivation for this work is to identify the association between the decoder state and the input word. This attention method aims to measure the impact of words representation in the source sentence by looking at the strength of this association to produce the subsequent output word.

\subsection{Transformer}
Recurrent models such as RNN usually compute each token symbol of the input and produce and input sequentially. During these sequential computations, they generate hidden states $h_t$ as a representation of the current position $t$. These hidden states take into consideration a combination of current input and the previous hidden states $h_{t-1}$. This behaviour implicitly forces the network to behave in a sequential manner and prevent parallelization within the training procedure. This parallelization becomes very important as the length of the input grows. Several works through factorization tricks \cite{Kuchaiev2017FactorizationTF} and conditional computation \cite{Shazeer2017OutrageouslyLN} have achieved notable improvements in reducing the computational time. However, it does not change the fact that the inherent sequential nature of the model remains.

To alleviate the sequential problem, \cite{vaswani2017attention} propose a new architecture called transformer. This new architecture avoids the recurrent nature of RNN entirely and only relies on  attention mechanism to provide dependencies between input and output. It allows more parallelization and reduces significant training time to achieve a new state of the art in several tasks such as machine translation.

We can refer to Figure \ref{img:transformer} for illustration of the transformer architecture. The architecture consists of two main components, encoder and decoder. The attention on the encoder side assigns an attention score to each word in the source sentence. The authors claim that compared to the sequential models, transformer is able to carry information between any pair of words in a single step and help the model make better performance while improving the training speed. The authors further propose an additional attention layer on the decoder that refers to the representation in the encoder for better context. This is helpful for tasks such as machine translation, where context from the source side is essential for the prediction.

\begin{figure}[h]
    {\includegraphics[width=0.75\textwidth]{img/transformer.png}}
    \centering
    \caption{Illustration of Transformer model. Figure reprinted from \cite{vaswani2017attention}.}
    \label{img:transformer}
\end{figure}

Based on \cite{liu2020understanding}, despite its contribution in leading many breakthroughs in NLP space, transformer requires non-trivial efforts in training the models. In contrary to other neural layers such as recurrent neural network (RNN) and convolution neural network (CNN), optimization such as stochastic gradient descent (SGD) may converge to bad/suspicious local optima if not tuned carefully. Furthermore, the warmup stage is crucial during the training as removing them leads to severe consequences such as model divergence. We understand from this finding that training transformer and obtaining an optimal performance is not straightforward.

\section{Transfer Learning}
\label{sec:bm_tl}
Transfer learning (TL) focuses on transferring knowledge from one problem to a different but related problem. Transfer learning involves a source domain $D_S$ and corresponding task $T_S$, a target domain $D_T$ and learning task $T_T$. We aim to learn and improve the target conditional distribution $P(Y_T|X_T)$ from $D_T$ by leveraging information from $D_S$ as well as $T_S$, where $D_S \neq D_T$, or $T_S \neq T_T$. For better illustration on TL, we can refer to Figure \ref{img:transfer_learning}.

\begin{figure}[h]
    {\includegraphics[width=0.95\textwidth]{img/transfer_learning_scenario.png}}
    \centering
    \caption{An illustration of transfer learning in different domain. Figure reprinted from \cite{ruder2019transfer}.}
    \label{img:transfer_learning}
\end{figure}

\cite{shavlik2010transfer} describe three ways of how transfer learning can
improve performance. Specifically:
\begin{itemize}
    \item Improving the initial performance at the beginning of training compared
          to a randomly initialized model when the tasks are similar;
    \item Shortening the time needed to reach the maximal performance;
    \item Improving the final performance level compared to training the model
          without the transfer
\end{itemize}

To some extent, we can see transfer learning as a way to initialize neural networks with more constraints than the usual definition of weights initialization. In weight initializations, we focus on initializing random weights for any type of neural network architecture. On the other hand, transfer learning is only applicable to a specific part of the neural network architecture within the same domain problem. Domain problem can be defined as a category of problems such as Computer Vision, Natural Language Processing, or Speech Recognition.
% As of this writing, we are not aware of any algorithm to perform transfer learning in different domain categories.

For this work, we are specifically interested in two of the following categories of transfer learning: 1) Domain adaptation; 2) Sequential transfer learning. In the following sections, we will discuss the difference between these two categories.

\subsection{Domain Adaptation}
\label{sec:domain_adapt}
In the context of NMT, we can distinguish two categories of transfer learning. The first category is domain adaptation, where we deal with the same language pairs but in different domains. For example, we pre-train a model in WMT data in German$\rightarrow$English pair and adapt IWSLT data within the same language. In the second category, we have multilingual adaptation, where we work with entirely different language pairs between the pre-training and the fine-tuning. For the scope of this project, we limit the problem to domain adaptation and only a single language pair.

The goal of domain adaptation is to optimize a model in a more specific domain. Models that are optimized on a specific genre (news, speech, medical, literature, and other) have higher accuracy than a system that is optimized for a more generic domain (\cite{gao2002improving,hildebrand2005adaptation}). This is due to the model's bias over the target domain. When the training data's distribution is unbiased towards the test set in a particular target domain, we expect to have similar performance compared to the training data. On the other hand, the performance will decline if the training data distribution is different.

In NMT, adaptation setup involves training models over two different data distribution (\cite{luong2015stanford,Servan2016DomainSA,Chu2018ASO}). The models are first trained on an out-of-domain parallel corpus containing broad information. More in-domain training data is introduced to fine-tune the model when the first training has finished. We can see this as a form of transfer learning where the gained knowledge from the out-of-domain corpus is leveraged by the model while fine-tuning in the in-domain corpus.

There are two problems in domain adaptation found by \cite{Freitag2016FastDA}:
\begin{itemize}
    \item The models are prone to over-fitting when the number of in-domain data is limited.
    \item The models are suffering from catastrophic forgetting when the models are fine-tuned. This means that the models performance in the out-of-domain data will degrade while the performance on in-domain data may be improved.
\end{itemize}
A proposed solution from \cite{Chu2017AnEC} address these problems by mixed fine-tuning. Essentially, they combine the out-of-domain corpus and in-domain data before adapting the general model.

\subsection{Sequential Transfer Learning}
Sequential transfer learning is a form of transfer learning that has led to the biggest improvements in NLP so far. In practice, we aim to perform a pre-training to build decent vector representations from a large unlabelled text corpus and then adapt these representations to a target task with labelled data. For better illustration, we refer to Figure \ref{img:seq_tl}

\begin{figure}[h]
    {\includegraphics[width=0.95\textwidth]{img/sequential_tl.png}}
    \centering
    \caption{Illustration of sequential transfer learning. Figure reprinted from \cite{ruder2019transfer}.}
    \label{img:seq_tl}
\end{figure}

In NLP, one of the most prominent examples of sequential transfer learning is pre-training with a language model objective. Language model pre-training has been shown as an effective objective in improving various NLP tasks (\cite{Dai2015SemisupervisedSL,Peters2018DeepCW,Radford2018ImprovingLU,Howard2018UniversalLM}). This includes sentence-level tasks in natural language understanding (NLU) such as natural language inference (\cite{Bowman2015ALA,Williams2018ABC}) and sentence paraphrasing \cite{Dolan2005AutomaticallyCA}. It also has been shown to improve performance in token-level tasks where models are expected to output another token, such as named entity recognition, question answering, and machine translation (\cite{Sang2003IntroductionTT,Rajpurkar2016SQuAD1Q}). In machine translation, the availability of high-quality parallel data can be a limitation to training good NMT models that can generate good output. Contextual knowledge such as the one from pre-trained models could be a good complement for NMT.

Although the language model task looks straightforward from a high-level overview, it is challenging for both machines and humans. For models to provide a solution, they must understand linguistics phenomena such as syntax and semantics as well as particular knowledge about the world. It has been shown that given enough data, enough computational power, and a large number of parameters, a model can provide a reasonable output \cite{radford2018improving}. Several works have shown, empirically, that language modelling performs better than other pre-training tasks such as translation or autoencoding \cite{Zhang2018LanguageMT,Wang2019CanYT}.

Based on \cite{ruder2019transfer}, there are two existing strategies for applying pre-trained representations: feature-based and fine-tuning. The feature-based approach works by incorporating the representation as an additional feature to the models in the downstream tasks. The example of this approach can be found in ELMo \cite{Peters2018DeepCW}. On the other hand, the fine-tuning approach employs the previously trained weights on the same model architecture in the downstream tasks. Several works such as BERT \cite{devlin2018bert} and OpenAI GPT \cite{Radford2018ImprovingLU} show fine-tuning provides a significant improvement. For that reason, we only consider the fine-tuning approach to train our model for this thesis.

\subsubsection{BERT}
This section discusses BERT as one of the most prominent pre-training models in NLP. It was proposed by \cite{devlin2018bert} as they argue that the current language model objective restricts the power of the pre-trained representations, especially when applied for fine-tuning in the downstream tasks. There is a significant limitation from the conditional language models where it only performs unidirectional prediction. For example, we can not use RNN in left-to-right and right-to-left (bidirectional RNN) directions simultaneously, as each direction will provide answers for the other.
This restriction can be harmful and sub-optimal for sentence-level tasks. Especially in machine translation, it would be more beneficial to encode the sentence in both directions on the encoder part as we are only concerned with obtaining a good representation from the source sentence.
Further illustration can be seen on \ref{img:bert}

\begin{figure}[h]
    {\includegraphics[width=0.95\textwidth]{img/bert.png}}
    \centering
    \caption{Illustration of BERT framework. Figure reprinted from \cite{devlin2018bert}.}
    \label{img:bert}
\end{figure}

Despite its success in many tasks, especially in NLU, incorporating BERT in natural language generation (NLG) remains challenging. There are several challenges in incorporating BERT in the sequence-to-sequence framework. \cite{Zhu2020IncorporatingBI} found that using pre-trained BERT as the initialization on the encoder side hurt the performance. An explanation for this fine-tuning BERT on a complex task requires extra care and may lead to the catastrophic forgetting problem \cite{mccloskey1989catastrophic} of the pre-trained model. On the decoder side, there is a mismatch in initializing the component with BERT due to the conditional nature of the training objective. We understand that we can treat the objective of machine translation in the decoder as a conditional language model. This is different from BERT as it uses a bidirectional objective such as Masked Language Model (MLM). Furthermore, fine-tuning the full weights of the model is inefficient considering the enormous amount of parameters within BERT. It is also tricky to fine-tune BERT in small datasets as the process can be unstable and fragile \cite{Lee2020MixoutER}.

\section{Adapters}
\label{sec:bm_adapters}
Adapter is a lightweight layer transplanted between the layers of a pre-trained transformer network and it is fine-tuned on the adaptation corpus. Adapter was proposed by \cite{houlsby2019parameter} as an alternative to fine-tuning. In this work, we follow the architecture similar to the work of \cite{bapna2019simple} as it is simpler to implement and has been adopted in other works (\cite{pfeiffer2020madx,ruckle2020adapterdrop,pfeiffer2021adapterfusion}). Specifically, we follow the adapter implementation of \cite{pfeiffer2020madx} as they have already performed an extensive hyperparameter search to find the best combination of adapter positions, activation functions, and residual connections within each transformer layer.

Following \cite{pfeiffer2020madx}, adapter is defined with the following formulation

$$A_l(h_l, r_l) = U_l(ReLU(D_l(LA_l))) + r_l $$

$A_l$ is the adapter incorporated at layer $l$, $D_l$ is a down projection layer $D \in R^{h \times d}$ where $h$ is the dimension of the current layer and $d$ is the adapter dimension, $U_l$ is an up projection layer $U \in R^{d \times h}$, and $r_l$ is the residual connection from the previous layer. We can refer to \ref{img:adapters} for illustration of the adapter bottleneck layer and how they are incorporated to the Transformer architecture.

\begin{figure}[h]
    {\includegraphics[width=0.75\textwidth]{img/adapter_module.png}}
    \centering
    \caption{Illustration of Adapters.}
    \label{img:adapters}
\end{figure}

There are several problems in fine-tuning that adapters are trying to solve:
\begin{itemize}
    \item The number of parameters in the state-of-the-art NMT has been increasing (\cite{Shazeer2018MeshTensorFlowDL,Bapna2018TrainingDN,Huang2019GPipeET}), and performing fine-tuning on all parameters is too costly.
    \item Full fine-tuning demands meticulous hyper-parameter tuning during the adaptation process, and it is prone to over-fitting (\cite{Sennrich2016ImprovingNM,Barone2017RegularizationTF}).
    \item \cite{Lee2020MixoutER} suggests that catastrophic forgetting leads to instability during fine-tuning.
    \item \cite{Mosbach2021OnTS} suggests gradient vanishing also contributes in instability during fine-tuning.
    \item The sensitivity to both hyper-parameter and over-fitting are intensified in the high capacity model.
\end{itemize}

\cite{han2021robust} shows that fixing the pre-trained layers and only fine-tune adapter modules improve the model's performance stability on various random seeds, enhance adversarial robustness, as well as better transfer learning performance. We can see from \cref{img:adapters_instability} the difference between models that were trained with (cluster on the right) and without adapters (cluster on the left) on different pre-training and fine-tuning iterations. The fine-tuned models with adapters show more robustness towards various pre-training models than the ones that were not using adapters.

\begin{figure}[h]
    {\includegraphics[width=0.95\textwidth]{img/adapters_instability.png}}
    \centering
    \caption{Illustration of Adapters instability. W represents the models that use adapters and WO represents the models that do not use adapters. Figure reprinted from \cite{han2021robust}.}
    \label{img:adapters_instability}
\end{figure}

% \subsection{Base model}
% Most works in adapters \cite{something} rely on BERT as their base models where all the weights are kept intact and only adapters are fine-tuned on the downstream tasks. Predominantly, the epxeriments are performed with pre-trained BERT weights that has been published publicly. Unfortunately, less study 
% The fundamental operation within deep neural network is matrix multiplication. A forward pass means performing sequential matrix multiplication starting from the input up to the last layer in deep neural network. Similar to forward pass, backward pass also performs matrix multiplication but in the opposite direction. In deep neural networks, we operate with multiple hidden layers as well as often thousands or more weights within a single layer. Performing matrix multiplication, with sub-optimal values may lead to sub-optimal results. To achieve optimal results, a good weight initialization is necessary to prevent faulty computation, specifically during backward operation.

% We recall from Section \ref{sec:adapters} that sequential transfer learning suffers from instability caused by gradient vanishing. 

% The aim of weight initialization is to prevent layer activation outputs from exploding or vanishing during the course of a forward pass through a deep neural network. If either occurs, loss gradients will either be too large or too small to flow backwards beneficially, and the network will take longer to converge, if it is even able to do so at all.

% Matrix multiplication is the essential math operation of a neural network. In deep neural nets with several layers, one forward pass simply entails performing consecutive matrix multiplications at each layer, between that layer's inputs and weight matrix. The product of this multiplication at one layer becomes the inputs of the subsequent layer, and so on and so forth.

% Most of the recent experimental results with deep architecture are obtained with models that can be turned into deep supervised neural networks, but with initialization or training schemes different from the classical feedforward neural networks (Rumelhart et al., 1986). Why are these new algorithms working so much better than the standard random initialization and gradient-based optimization of a supervised training criterion? Part of the answer may be found in recent analyses of the effect of unsupervised pretraining (\cite{erhan2009thedifficulty}), showing that it acts as a regularizer that initializes the parameters in a “better” basin of attraction of the optimization procedure, corresponding to an apparent local minimum associated with better generalization.

% \begin{figure}[h]
%     {\includegraphics[width=0.95\textwidth]{img/comp_init.png}}
%     \centering
%     \caption{Comparison of neural network performance using different initialization technique. Figure reprinted from \cite{kumar2017onweight}.}
%     \label{img:comp_init}
% \end{figure}

% To show the impact of good initialization, we refer to the work of \cite{kumar2017onweight}. In this work, the author provides a recommendation to initialize a neural network with sigmoid activation function. The experiment was done in Computer Vision area using CIFAR10 dataset. The author compares the result of the initialization with one of the most used initialization such as Xavier initialization (\cite{glorot2010understanding}). The result of the experiment can be seen at \ref{img:comp_init}. We can see from this graph the gap between different initialization is quite apparent.