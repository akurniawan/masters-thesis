%%% The main file. It contains definitions of basic parameters and includes all other parts.

%% Settings for single-side (simplex) printing
% Margins: left 40mm, right 25mm, top and bottom 25mm
% (but beware, LaTeX adds 1in implicitly)
\documentclass[12pt,a4paper]{report}
\setlength\textwidth{145mm}
\setlength\textheight{247mm}
\setlength\oddsidemargin{15mm}
\setlength\evensidemargin{15mm}
\setlength\topmargin{0mm}
\setlength\headsep{0mm}
\setlength\headheight{0mm}
% \openright makes the following text appear on a right-hand page
\let\openright=\clearpage

%% Settings for two-sided (duplex) printing
% \documentclass[12pt,a4paper,twoside,openright]{report}
% \setlength\textwidth{145mm}
% \setlength\textheight{247mm}
% \setlength\oddsidemargin{14.2mm}
% \setlength\evensidemargin{0mm}
% \setlength\topmargin{0mm}
% \setlength\headsep{0mm}
% \setlength\headheight{0mm}
% \let\openright=\cleardoublepage

%% Generate PDF/A-2u
\usepackage[a-2u]{pdfx}

%% Character encoding: usually latin2, cp1250 or utf8:
\usepackage[utf8]{inputenc}

%% Prefer Latin Modern fonts
\usepackage{lmodern}

%% Further useful packages (included in most LaTeX distributions)
\usepackage{amsmath}        % extensions for typesetting of math
\usepackage{amsfonts}       % math fonts
\usepackage{amsthm}         % theorems, definitions, etc.
\usepackage{bbding}         % various symbols (squares, asterisks, scissors, ...)
\usepackage{bm}             % boldface symbols (\bm)
\usepackage{graphicx}       % embedding of pictures
\usepackage{fancyvrb}       % improved verbatim environment
\usepackage{natbib}         % citation style AUTHOR (YEAR), or AUTHOR [NUMBER]
\usepackage[nottoc]{tocbibind} % makes sure that bibliography and the lists
			    % of figures/tables are included in the table
			    % of contents
\usepackage{dcolumn}        % improved alignment of table columns
\usepackage{booktabs}       % improved horizontal lines in tables
\usepackage{paralist}       % improved enumerate and itemize
\usepackage{xcolor}         % typesetting in color
\usepackage{natbib}
\usepackage{url}
\usepackage{rotating}
\usepackage{multirow}
\usepackage[capitalize,noabbrev]{cleveref}

%%% Basic information on the thesis

% Thesis title in English (exactly as in the formal assignment)
\def\ThesisTitle{Adapting Pretrained Models for Machine Translation}

% Author of the thesis
\def\ThesisAuthor{Aditya Kurniawan}

% Year when the thesis is submitted
\def\YearSubmitted{2022}

% Name of the department or institute, where the work was officially assigned
% (according to the Organizational Structure of MFF UK in English,
% or a full name of a department outside MFF)
\def\Department{Institute of Formal and Applied Linguistics}

% Is it a department (katedra), or an institute (ústav)?
\def\DeptType{Institute}

% Thesis supervisor: name, surname and titles
\def\Supervisor{doc. RNDr. Ond\r{v}ej Bojar, Ph.D.}

% Supervisor's department (again according to Organizational structure of MFF)
\def\SupervisorsDepartment{Institute of Formal and Applied Linguistics}

% Study programme and specialization
\def\StudyProgramme{Informatics}
\def\StudyBranch{Language Technologies and Computational Linguistics}

% An optional dedication: you can thank whomever you wish (your supervisor,
% consultant, a person who lent the software, etc.)
\def\Dedication{%
First and foremost, I would like to express my sincere gratitude to my supervisors Doc. RNDR. Ond\r{v}ej Bojar, Ph.D from Charles University and Dr Marc Tanti from University of Malta for their overall support, thorough feedback, and guidance during the creation of this thesis.

Secondly, I would like to thank the Erasmus Mundus European Master Program in Language and Communication Technologies (LCT) for the scholarship that allowed me to go through this two-year journey. I am also very thankful to Prof. Markéta Lopatková and Prof. Vladislav Kuboň from Charles University; Prof. Lonneke van Der Plas and Prof. Claudia Borg from the University of Malta; and Dr Bobbye Pernice and Anna Felsing from Universität des Saarlandes for their colossal help in this entire period of study.

Thirdly, I thank MetaCentrum for providing the resources required to complete this thesis. We used the computational resources supplied by the project ``e-Infrastruktura CZ'' (e-INFRA CZ LM2018140), supported by the Ministry of Education, Youth and Sports of the Czech Republic.

Finally, I would also like to thank my family and friends for their support throughout my study.
}

% Abstract (recommended length around 80-200 words; this is not a copy of your thesis assignment!)
\def\Abstract{%
Pre-trained language models received extensive attention in recent years. However, it is still challenging to incorporate a pre-trained model such as BERT into natural language generation tasks. This work investigates a recent method called adapters as an alternative to fine-tuning the whole model in machine translation. Adapters are a promising approach that allows fine-tuning only a tiny fraction of a pre-trained network.
We show that with proper initialization, adapters can help achieve better performance than training models from scratch while training substantially fewer weights than the original model.
We further show that even with randomly set weights used as the base models for fine-tuning, we can achieve similar performance to one of the baseline models, bypassing the need to train hundreds of millions of weights in the pre-training phase.
Furthermore, we study the effectiveness of adapters in the Transformer model for machine translation. We put adapters either in the encoder or the decoder only, and we also attempt to down-scale the pre-trained model size to decrease GPU memory demands.
We found that incorporating adapters in the encoder alone matches the setup's performance when we include the adapters on both the encoder and decoder.
Finally, our down-scaling study found that using only half of the original pre-trained weights can positively impact the performance when fine-tuned with adapters.
}

% 3 to 5 keywords (recommended), each enclosed in curly braces
\def\Keywords{%
% {key} {words}
{adapters} {machine translation} {bert} {transformer} {transfer learning}
}

%% The hyperref package for clickable links in PDF and also for storing
%% metadata to PDF (including the table of contents).
%% Most settings are pre-set by the pdfx package.
\hypersetup{unicode}
\hypersetup{breaklinks=true}

% Definitions of macros (see description inside)
\include{macros}

% Title page and various mandatory informational pages
\begin{document}
\include{title}

%%% A page with automatically generated table of contents of the master thesis

\tableofcontents

%%% Each chapter is kept in a separate file
\chapter*{Introduction}
\addcontentsline{toc}{chapter}{Introduction}
Pre-trained language models \citewithpar{devlin2018bert,howard2018universal} have received considerable attention in recent years. These models are trained on large-scale corpora and then fine-tuned for a particular downstream task. This method allows pre-trained models to perform well across a range of natural language processing tasks. One of the most successful models is BERT \citewithpar{devlin2018bert}. BERT has been most extensively used for common natural language understanding (NLU) tasks. It has been shown that BERT can achieve great performance with relatively straightforward fine-tuning, especially for classification-like tasks.

For natural language generation (NLG), incorporating BERT is still challenging. According to \normcite{zhu2020incorporating}, simply incorporating BERT into the encoder side of the seq2seq architecture can hurt the performance. On the decoder side, BERT does not quite fit either because the bidirectional nature of the model was significantly different from the conditional language model (predicting the next word) objective we are aiming for.

Fine-tuning all BERT's parameters is inefficient given that there are approximately 200 million parameters in a single model of BERT. Naive fine-tuning also often results in catastrophic forgetting, where the models forget the previous knowledge they have acquired while improving on the new domain \citewithpar{mccloskey1989catastrophic,yogatama2019learning}. This may explain why it is considered harmful to simply fine-tune an initialized encoder component with BERT. It is also known that fine-tuning large pre-trained language models could result in unstable and fragile performance on small datasets.

Adapter is an alternative approach that allows for fine-tuning a model without altering the original network weights \citewithpar{houlsby2019parameter,bapna2019simple}. By leveraging adapters, one can reduce the number of parameters updated in fine-tuning and make the process computationally less expensive while achieving similar results. Another useful property of the approach with adapters is that they are more robust against catastrophic forgetting than fine-tuning \citewithpar{han2021robust}.

This work uses BERT and its variants as the base pre-trained models and fine-tune them with adapters. We evaluate the models on machine translation with the following objectives:
\begin{itemize}
    \item We conduct a study to understand the contribution of good representation in the pre-trained language model when fine-tuning using adapters.
    \item We conduct a study to evaluate the effectiveness of adapters in the seq2seq framework by putting them only in the encoder or the decoder.
    \item We experiment with down-scaling the pre-trained model size and try to recover the performance from being comparable to the full-sized model.
\end{itemize}

\section*{Thesis Organization}

\paragraph{Chapter 1} discusses the theoretical background of machine translation, transfer learning, and a brief overview of the current state of using adapters in various setups.

\paragraph{Chapter 2} reviews the previous related work on transfer learning from models that were pre-trained on language model objectives and the usage of adapters in various disciplines within text and speech domains.

\paragraph{Chapter 3} describes the dataset that we use to train language models and machine translation. We then explain the pre-processing of the dataset and the tokenization to construct the vocabularies. Finally, we describe the framework for the experiment and the automatic evaluation metric.

\paragraph{Chapter 4} presents our attempt to use adapters in machine translation setup. This chapter mainly focuses on the contribution of the pre-trained representation during fine-tuning with adapters in machine translation. We then provide discussions of our findings in more detail.

\paragraph{Chapter 5} presents our attempt to understand the effectiveness of adapters as well as the impact of the pre-trained weights for adapters by placing them only in either the encoder or decoder. We then continue the experiments by down-scaling BERT to half of the size and trying to recover the adapter's performance so that it is comparable to the full-sized model. We provide discussions of the phenomenon that happens when reducing the size of the original BERT model.

\paragraph{Conclusion} summarizes our findings from the experiments.
\chapter{Background and Motivation}

In this chapter, we summarize the background and motivation underlying this work. In Section \ref{sec:bm_smt}, we briefly review the approaches to machine translation (MT) and the comparison between them.

Section \ref{sec:bm_nmt} reviews the formulation of neural machine translation as well as the frameworks utilized to solve the problem.

% Section \ref{sec:weight_init} defines weight initialization in deep neural network and the importance of finding the proper initialization to be able to gain optimum performance.

Section \ref{sec:bm_tl} discuss transfer learning (TL) as well as its variants.

Finally, Section \ref{sec:bm_adapters} discuss the adapter module as an alternative methodology in fine-tuning the model and provides reasons why we focus our research by using adapters.

\section{Machine Translation}
\label{sec:bm_smt}
On a fundamental level, MT performs the substitution of words from one language to another. However, it is challenging to produce a good translation based on the substitution alone as understanding the whole sentence that includes phrases and surrounding words in the target language is needed. The problem is exacerbated as words may have more than one meaning, and it is difficult to determine one-to-one relations in another language.

The three most commonly used approaches in MT are rule-based, statistical (SMT) and neural (NMT). Due to the significant effort in manually collecting good dictionary and grammatical rules, demands on a more automatic approach such as SMT or NMT seems more appealing.
Before NMT, a variant of SMT, namely phrase-based machine translation (PBMT), had been the state-of-the-art for German-to-English language pairs. \cite{bojar2015proceeding} also shows that PBMT has a good performance in different language pairs. \cite{blunsom2013recurrent} introduced the first end-to-end neural network for machine translation with an encoder-decoder structure. Their approach encodes a given source text into a continuous vector and further transform the state vector into the target language.

While NMT has been the primary technique used in various MT challenges, such as WMT, there are some advantages and disadvantages. According to \cite{koehn2017nmt}, NMT suffers from the following phenomenon:
\begin{itemize}
    \item In an out-of-domain scenario, NMT systems have a lower quality. The author found that the model chooses to sacrifice adequacy over fluency completely.

    \item NMT requires a large amount of data. It is problematic when low-resource languages are involved in the evaluation.

    \item PBMT performance suffers when low-frequency words occur in the sentence. It is especially true when the word is entirely unknown. Although NMT is better performed in low-frequency words, the problem is not yet solved. The NMT models also show weakness in translating low-frequency words from the experiments.

    \item Difficulties in translating long sentences. NMT can perform well in short sentences up to 60 words. However, longer sentences show a lower quality of translation.

    \item Model interpretability. As opposed to PBMT, it is not easy to interpret the behaviour of NMT due to the complexity introduced in the hyperparameter and the model architecture. Furthermore, the training of NMT is also non-deterministic due to random parameter initialization.
\end{itemize}

Despite its shortcomings, NMT also shows a promising direction in MT. \cite{machavcek2018enriching} mentioned that the difference is apparent in the output fluency. They mentioned that PBMT models suffer from double negation and translation morphologically rich languages. These problems cause little to no problem at all in NMT models.

\section{Neural Machine Translation}
\label{sec:bm_nmt}
We define the translation problem as a mapping function $t$ of sentences from a given source language $S$ and target language $T$ from parallel corpus, where $t : S \rightarrow T$. A parallel corpus is a pair of sentences in two different languages where one sentence in $T$ corresponds to its equivalent in language $S$. The goal of function $t$ is to find the highest probability of word $y \in T$ from $x \in S$, where $t(x) = argmax_y(p(y|x))$. The probability $p(y|x)$ is the probability estimation given by the NMT model.

The following sections show a quick overview of recent NMT models. They include architecture, advantages, and drawbacks.

\subsection{RNN Seq2Seq}
The first seq2seq model that employs RNN as the fundamental architecture is proposed by \cite{sutskever2014sequence}. This is a straightforward extension of language model problem. Essentially, the model sequentially predicts the next word given all previous words.

In MT, the approach is modified by using two similar model architectures for language $S$ and $T$. For language $S$ we call this component \texttt{encoder} and for $T$ we call it \texttt{decoder}.
The task of the \texttt{encoder} is to produce a vector representation of the input sentence from source language $x \in S$. We define the input sentence as a sequence of tokens from a fixed set of vocabulary $x \in S$ transformed by an embedding matrix containing dense representation. An RNN is then used to process these representations. This results in a new representation for each token encoded by hidden states in RNN. This representation can be thought of as a combination of features from the token and its context.
\texttt{decoder} has a similar functionality as the \texttt{encoder} where it uses a sequence of tokens from $y \in T$ as its inputs. The \texttt{decoder} leverage additional features from the \texttt{encoder} by incorporating the output vector of the \texttt{encoder}. The output vector from the \texttt{encoder} represents the final token of the sentence. For further illustration, we refer to Image \ref{img:rnnseq2seq}.

\begin{figure}[h]
    {\includegraphics[width=0.95\textwidth]{img/rnnseq2seq.png}}
    \centering
    \caption{Illustration of seq2seq architecture\protect\footnotemark[1].}
    \label{img:rnnseq2seq}
\end{figure}

The disadvantages of this model are found by \cite{cho2014properties}. They found that the models' performance decreased when the length of the source sentence increased. We recall that the only features used by the \texttt{decoder} to refer to the source sentence are through the last token vector from the \texttt{encoder}. The vector is a fixed-size vector with a pre-defined length prior to the training. In essence, this vector tries to combine the features from all the words in the source sentence. Hence, when the source sentence grows, the vector could be less informative due to the more information it has to encode.

\footnotetext[1]{Figure reprinted from \protect\url{https://www.guru99.com/seq2seq-model.html}.}

Furthermore, RNNs also suffer where the gradient can be extremely small or large. These problems are often mentioned as vanishing and exploding gradient problems. When the model's gradient is extremely small, RNNs can not learn from the data effectively, especially in the long-range dependencies setup. On the other hand, when the gradient is extremely large, it can affect the weight parameters by moving them far away from the optimal space. This would disrupt the learning process and cause the model to fail to learn. This problem can happen in seq2seq architecture as we essentially backpropagate the weights from the end of the \texttt{decoder} to the beginning of the \texttt{encoder}.

\subsection{Seq2Seq with Attention}
An extension of the encoder-decoder model is proposed by jointly learning to translate and align \cite{bahdanau2015nmt}. This method learns to identify the most relevant sources of information in a source sentence. It then proceeds to use the context vectors associated with the source positions to predict a target word.

\begin{figure}[h]
    {\includegraphics[width=0.95\textwidth]{img/attseq2seq.png}}
    \centering
    \caption{Illustration of seq2seq architecture with attention\protect\footnotemark[2].}
    \label{img:attseq2seq}
\end{figure}

This method eliminates the need for a neural model to learn to squash the whole sentence into a fixed-length vector. Instead of trying to encode a whole sentence as the method proposed by \cite{sutskever2014sequence}, it selects a subset of vectors from the source sentence that is deemed to contain the most relevant information to be used while decoding the translated message. This proves to allow the model to provide better predictions in long sentences. We can see the illustration of this architecture from Figure \ref{img:attseq2seq}.

\footnotetext[2]{Figure reprinted from \protect\url{https://lena-voita.github.io/nlp_course/seq2seq_and_attention.html}.}

The proposed approach achieves significantly better translation performance than the basic encoder-decoder model. The improvement is especially evident in longer sentences. The model shows comparable performance to or close to the phrase-based system in the En-Fr pair.

The motivation for this work is to identify the association between the decoder state and the input word. This attention method wants to measure the impact of words representation in the source sentence by looking at the strength of this association to produce the subsequent output word.

\subsection{Transformer}
Recurrent models such as RNN usually compute each token symbol of the input and produce and input sequentially. During these sequential computations, they generate hidden states $h_t$ as a representation of the current position $t$. These hidden states take into consideration a combination of current input and the previous hidden states $h_{t-1}$. This behaviour implicitly forces the network to behave in a sequential manner and prevent parallelization within the training procedure. This parallelization becomes very important as the length of the input grows. Several works through factorization tricks \cite{Kuchaiev2017FactorizationTF} and conditional computation \cite{Shazeer2017OutrageouslyLN} have achieved notable improvements in reducing the computational time. However, it does not change the fact that the inherent sequential nature of the model remains.

To alleviate the sequential problem, \cite{vaswani2017attention} propose a new architecture called transformer. This new architecture avoids the recurrent nature of RNN entirely and only relies on an attention mechanism to provide dependencies between input and output. It allows more parallelization and reduces significant training time to achieve a new state of the art in several tasks. MT is one of them.

We can refer to Figure \ref{img:transformer} for illustration of the transformer architecture. The architecture consists of two main components, encoder and decoder. The attention on the encoder side assigns an attention score to each word in the source sentence. The authors claim that compared to the sequential models, transformer is able to transport information between any pair of words in a single step and help the model make better performance and improve the training speed. There is an additional attention layer on the decoder that refers to the representation in the encoder for better context. This is helpful for tasks such as machine translation, where context from the source side is essential for the prediction.

\begin{figure}[h]
    {\includegraphics[width=0.75\textwidth]{img/transformer.png}}
    \centering
    \caption{Illustration of Transformer model. Figure reprinted from \cite{vaswani2017attention}.}
    \label{img:transformer}
\end{figure}

Based on \cite{liu2020understanding}, despite its contribution in leading many breakthroughs in NLP space, transformer requires non-trivial efforts in training the models. In contrary to other neural layers such as recurrent neural network (RNN) and convolution neural network (CNN), optimization such as stochastic gradient descent (SGD) may converge to bad/suspicious local optima if not tuned carefully. Furthermore, the warmup stage is crucial during the training as removing them leads to severe consequences such as model divergence. We understand from this finding that training Transformer and obtaining an optimal performance is not straightforward.

\section{Transfer Learning}
\label{sec:bm_tl}
Transfer learning (TL) focuses on transferring knowledge from one problem to a different but related problem. Transfer learning involves a source domain $D_S$ and corresponding task $T_S$, a target domain $D_T$ and learning task $T_T$. We aim to learn and improve the target conditional distribution $P(Y_T|X_T)$ from $D_T$ by leveraging information from $D_S$ as well as $T_S$, where $D_S \neq D_T$, or $T_S \neq T_T$. For better illustration on TL, we can refer to Figure \ref{img:transfer_learning}.

\begin{figure}[h]
    {\includegraphics[width=0.95\textwidth]{img/transfer_learning_scenario.png}}
    \centering
    \caption{An illustration of transfer learning in different domain. Figure reprinted from \cite{ruder2019transfer}.}
    \label{img:transfer_learning}
\end{figure}

\cite{shavlik2010transfer} describe three ways of how transfer learning can
improve performance. Specifically:
\begin{itemize}
    \item improving the initial performance at the beginning of training compared
          to a randomly initialized model when the tasks are similar;
    \item shortening the time needed to reach the maximal performance;
    \item improving the final performance level compared to training the model
          without the transfer
\end{itemize}

To some extent, we can see transfer learning as a way to initialize neural networks with more constraints than the usual definition of weights initialization. In weight initializations, we focus on initializing random weights for any type of neural network architecture. On the other hand, transfer learning is only applicable to a specific part of the neural network architecture within the same domain problem. Domain problem can be defined as a category of cognitive problems such as Computer Vision, Natural Language Processing, or Speech Recognition. As of this writing, we are not aware of any algorithm to perform transfer learning in different domain categories.

For this work, we are specifically interested in two of the following categories of transfer learning: 1) Domain adaptation; 2) Sequential transfer learning. In the following sections, we will discuss the difference between these two categories.

\subsection{Domain Adaptation}
\label{sec:domain_adapt}
In the context of NMT, we can distinguish two categories of transfer learning. The first category is domain adaptation, where we are dealing with the same language pairs but in different domains. For example, we pre-train a model in WMT data in German$\rightarrow$English pair and adapt IWSLT data within the same language. In the second category, we have multilingual adaptation. In this category, we are dealing with entirely different language pairs between the pre-training and the fine-tuning. For this project's scope, we limit the problem to domain adaptation and only a single language pair.

The goal of domain adaptation is to optimize a model in a more specific domain. Models that are optimized on a specific genre (news, speech, medical, literature, and other) have higher accuracy than a system that is optimized for a more generic domain (\cite{gao2002improving,hildebrand2005adaptation}). This is due to the model's bias over the target domain. When the training data's distribution is unbiased towards the test set in a particular target domain, we expect we would have similar performance compared to the training data. On the other hand, the performance will decline if the training data distribution is different.

In NMT, adaptation setup involves training models over two different data distribution (\cite{luong2015stanford,Servan2016DomainSA,Chu2018ASO}). The models are first trained on an out-of-domain parallel corpus containing broad information. More in-domain training data is introduced to fine-tune the model when the first training has finished. We can see this as a form of transfer learning where the gained knowledge from the out-of-domain corpus is leveraged by the model while fine-tuning in the in-domain corpus.

There are two problems in domain adaptation found by \cite{Freitag2016FastDA}:
\begin{itemize}
    \item The models are prone to over-fitting when the number of in-domain data is limited.
    \item The models are suffering from catastrophic forgetting when the models are fine-tuned. This means that the models performance in the out-of-domain data will degrade while the performance on in-domain data may be improved.
\end{itemize}
A proposed solution from \cite{Chu2017AnEC} address these problems by mixed fine-tuning. Essentially, they combine the out-of-domain corpus and in-domain data before adapting the general model.

\subsection{Sequential Transfer Learning}
Sequential transfer learning is a form of transfer learning that has led to the biggest improvements on NLP so far. In practice, we aim to perform a pre-training to build decent dense vector representations from a large unlabelled text corpus and then adapt these representations in a target task using labelled data.For better illustration, we refer to Figure \ref{img:seq_tl}

\begin{figure}[h]
    {\includegraphics[width=0.95\textwidth]{img/sequential_tl.png}}
    \centering
    \caption{Illustration of sequential transfer learning. Figure reprinted from \cite{ruder2019transfer}.}
    \label{img:seq_tl}
\end{figure}

In NLP, one of the most prominent examples of sequential transfer learning is language model pre-training. Language model pre-training has been shown as an effective objective to improve many NLP tasks (\cite{Dai2015SemisupervisedSL,Peters2018DeepCW,Radford2018ImprovingLU,Howard2018UniversalLM}). These include sentence-level tasks in natural language understanding (NLU) such as natural language inference (\cite{Bowman2015ALA,Williams2018ABC}) and sentence paraphrasing \cite{Dolan2005AutomaticallyCA}. It also has been shown to improve performance in token-level tasks where models are expected to output another token, such as named entity recognition, question answering, and machine translation (\cite{Sang2003IntroductionTT,Rajpurkar2016SQuAD1Q}). In machine translation, the availability of high-quality parallel data can be a limitation to training good NMT models that can generate good output. Contextual knowledge such as the one from pre-trained models could be a good complement for NMT.

Although the language model task looks straightforward from a high-level overview, it is challenging for both machines and humans. For models to provide a solution, they must understand certain phenomena such as syntax, semantics, and particular knowledge about the world. It has been shown that given enough data, enough computational power, and a large number of parameters; a model can provide a reasonable output \cite{radford2018improving}. Several works have shown, empirically, that language modelling performs better than other pre-training tasks such as translation or autoencoding \cite{Zhang2018LanguageMT,Wang2019CanYT}.

Based on \cite{ruder2019transfer}, there are two existing strategies for applying pre-trained representations: feature-based and fine-tuning. The feature-based approach works by incorporating the representation as an additional feature to the models in the downstream tasks. The example of this approach can be seen in ELMo \cite{Peters2018DeepCW}. On the other hand, the fine-tuning approach employs the previously trained weights on the same model architecture in the downstream tasks. Several works show fine-tuning provides a significant improvement, such as BERT \cite{devlin2018bert} and OpenAI GPT \cite{Radford2018ImprovingLU}. We only consider the fine-tuning approach to train our model for this thesis.

\subsubsection{BERT}
This section discusses BERT as one of the most prominent pre-training algorithms in NLP. It was proposed by \cite{devlin2018bert} as they argue that current techniques restrict the power of the pre-trained representations, especially for the fine-tuning approaches. There is a significant limitation from the conditional language models where it only performs unidirectional prediction. For example, we can not use RNN in left-to-right and right-to-left (bidirectional RNN) directions simultaneously as each direction will answer the other.
This restriction can be harmful and sub-optimal for sentence-level tasks. Especially in machine translation, it would be more beneficial to encode the sentence in both directions on the encoder part as we are only concerned to obtain a good representation from the source sentence.
Further illustration can be seen on \ref{img:bert}

\begin{figure}[h]
    {\includegraphics[width=0.95\textwidth]{img/bert.png}}
    \centering
    \caption{Illustration of BERT framework. Figure reprinted from \cite{devlin2018bert}.}
    \label{img:bert}
\end{figure}

Despite its success in many tasks, especially in natural language understanding (NLU), incorporating BERT in natural language generation (NLG) remains challenging. There are several challenges in incorporating BERT in the sequence-to-sequence framework. \cite{Zhu2020IncorporatingBI} found that using pre-trained BERT as the initialization on the encoder side hurt the performance. An explanation for this fine-tuning BERT on a complex task requires extra care and may lead to the catastrophic forgetting problem \cite{mccloskey1989catastrophic} of the pre-trained model. On the decoder side, there is a mismatch in initializing the component with BERT due to the conditional nature of the training objective. We understand that we can treat the objective of machine translation in the decoder as a conditional language model. This is different from BERT as it uses a bidirectional objective such as Masked Language Model (MLM). Furthermore, fine-tuning the full weights of the model is inefficient considering the enormous amount of parameters within BERT. It is also tricky to fine-tune BERT in small datasets as the process can be unstable and fragile \cite{Lee2020MixoutER}.

\section{Adapters}
\label{sec:bm_adapters}
Adapters are a lightweight layer transplanted between the layers of a pre-trained Transformer network and fine-tuned on the adaptation corpus. Adapters were proposed by \cite{houlsby2019parameter} as an alternative to fine-tuning. There are two different adapter placements as proposed by \cite{bapna2019simple} and \cite{houlsby2019parameter}. The former leverages the adapters in two different parts of the sub-layers. The latter only appends the adapters on top of each layer with layer normalization added within the adapter architecture. As of this writing, there are no direct comparisons between these two techniques. However, the work of \cite{bapna2019simple} is simpler to implement and has been adopted in other works (\cite{pfeiffer2020madx,ruckle2020adapterdrop,pfeiffer2021adapterfusion}).

Following \cite{pfeiffer2020madx}, adapter is defined with the following formulation

$$A_l(h_l, r_l) = U_l(ReLU(D_l(LA_l))) + r_l $$

$A_l$ is the adapter incorporated at layer $l$, $D_l$ is a down projection layer $D \in R^{h \times d}$ where $h$ is the dimension of the current layer and $d$ is the adapter dimension, $U_l$ is an up projection layer $U \in R^{d \times h}$, and $r_l$ is the residual connection from the previous layer. We can refer to \ref{img:adapters} for illustration of the adapter bottleneck layer and how they are incorporated to the Transformer architecture.

\begin{figure}[h]
    {\includegraphics[width=0.75\textwidth]{img/adapter_module.png}}
    \centering
    \caption{Illustration of Adapters.}
    \label{img:adapters}
\end{figure}

There are several problems in fine-tuning that adapters are trying to solve:
\begin{itemize}
    \item The number of parameters in the state-of-the-art NMT has been increasing (\cite{Shazeer2018MeshTensorFlowDL,Bapna2018TrainingDN,Huang2019GPipeET}), and performing fine-tuning on all parameters is too costly.
    \item Full fine-tuning demands meticulous hyper-parameter tuning during the adaptation process, and it is prone to over-fitting (\cite{Sennrich2016ImprovingNM,Barone2017RegularizationTF}).
    \item \cite{Lee2020MixoutER} suggests that catastrophic forgetting leads to instability during fine-tuning.
    \item \cite{Mosbach2021OnTS} suggests gradient vanishing also contributes in instability during fine-tuning.
    \item The sensitivity to both hyper-parameter and over-fitting are intensified in the high capacity model.
\end{itemize}

\cite{han2021robust} shows that fixing the pre-trained layers and only fine-tune adapter modules improve the model's performance stability on various random seeds, enhance adversarial robustness, as well as better transfer learning performance. We can see from \cref{img:adapters_instability} the difference between models that were trained with (cluster on the right) and without adapters (cluster on the left) on different pre-training and fine-tuning iteration. The models that fine-tuned with adapters show more robustness towards variety of different pre-training models compared to the ont that were not using adapters.

\begin{figure}[h]
    {\includegraphics[width=0.95\textwidth]{img/adapters_instability.png}}
    \centering
    \caption{Illustration of Adapters instability. W represents the models that use adapters and WO represents the models that do not use adapters. Figure reprinted from \cite{han2021robust}.}
    \label{img:adapters_instability}
\end{figure}

% \subsection{Base model}
% Most works in adapters \cite{something} rely on BERT as their base models where all the weights are kept intact and only adapters are fine-tuned on the downstream tasks. Predominantly, the epxeriments are performed with pre-trained BERT weights that has been published publicly. Unfortunately, less study 
% The fundamental operation within deep neural network is matrix multiplication. A forward pass means performing sequential matrix multiplication starting from the input up to the last layer in deep neural network. Similar to forward pass, backward pass also performs matrix multiplication but in the opposite direction. In deep neural networks, we operate with multiple hidden layers as well as often thousands or more weights within a single layer. Performing matrix multiplication, with sub-optimal values may lead to sub-optimal results. To achieve optimal results, a good weight initialization is necessary to prevent faulty computation, specifically during backward operation.

% We recall from Section \ref{sec:adapters} that sequential transfer learning suffers from instability caused by gradient vanishing. 

% The aim of weight initialization is to prevent layer activation outputs from exploding or vanishing during the course of a forward pass through a deep neural network. If either occurs, loss gradients will either be too large or too small to flow backwards beneficially, and the network will take longer to converge, if it is even able to do so at all.

% Matrix multiplication is the essential math operation of a neural network. In deep neural nets with several layers, one forward pass simply entails performing consecutive matrix multiplications at each layer, between that layer's inputs and weight matrix. The product of this multiplication at one layer becomes the inputs of the subsequent layer, and so on and so forth.

% Most of the recent experimental results with deep architecture are obtained with models that can be turned into deep supervised neural networks, but with initialization or training schemes different from the classical feedforward neural networks (Rumelhart et al., 1986). Why are these new algorithms working so much better than the standard random initialization and gradient-based optimization of a supervised training criterion? Part of the answer may be found in recent analyses of the effect of unsupervised pretraining (\cite{erhan2009thedifficulty}), showing that it acts as a regularizer that initializes the parameters in a “better” basin of attraction of the optimization procedure, corresponding to an apparent local minimum associated with better generalization.

% \begin{figure}[h]
%     {\includegraphics[width=0.95\textwidth]{img/comp_init.png}}
%     \centering
%     \caption{Comparison of neural network performance using different initialization technique. Figure reprinted from \cite{kumar2017onweight}.}
%     \label{img:comp_init}
% \end{figure}

% To show the impact of good initialization, we refer to the work of \cite{kumar2017onweight}. In this work, the author provides a recommendation to initialize a neural network with sigmoid activation function. The experiment was done in Computer Vision area using CIFAR10 dataset. The author compares the result of the initialization with one of the most used initialization such as Xavier initialization (\cite{glorot2010understanding}). The result of the experiment can be seen at \ref{img:comp_init}. We can see from this graph the gap between different initialization is quite apparent.
\chapter{Related Work}

This chapter reviews the previous works related to this thesis on the adaptation of transformer and BERT with adapters in machine translation. Specifically, in \cref{sec:prelm_mt}, we review works in incorporating BERT for machine translation. In \cref{sec:adapter_seq}, we review the usage of adapters using transformer as the base model in sequence to sequence models in various fields such as NLU, Automatic Speech Recognition (ASR), and MT.

\section{Pre-training Language Models in Machine Translation}
\label{sec:prelm_mt}

This section will discuss three different works that try to incorporate BERT into NMT. The goals of these works are different from the experiments proposed in this thesis because they are primarily leveraging the feature representation of BERT instead of using BERT directly to fine-tune NMT.

\cite{weng2020acquiring} propose to acquire knowledge from pre-trained models such as BERT with a framework called $A_{PT}$. $A_{PT}$ consist of two different modules with different goals. The first goal is to learn a task-specific representation through adaptation from general representation in the pre-trained models and to learn two controlling methods to control the task-specific representation into NMT. They propose to achieve the first goal by using a dynamic fusion mechanism. They claim this method could provide rich contextual information to model sentences in NMT better. An illustration of their proposed approach can be seen in \cref{img:dyn_fn}. The second goal is via knowledge distillation to prune the knowledge from pre-trained models. They claim this method could help NMT to continuously learn essential knowledge about translation from source sentence to target sentence by using parallel data and help generate better translation by learning from monolingual data. Moreover, based on their empirical results, they achieve the best results when both the methods are applied in the encoder and decoder. We refer to \cref{img:kdweng} for illustration of this approach.

\begin{figure}[h]
    {\includegraphics[width=0.75\textwidth]{img/dynamic_fusion.png}}
    \centering
    \caption{Illustration of dynamic fusion mechanism. Figure reprinted from \cite{weng2020acquiring}.}
    \label{img:dyn_fn}
\end{figure}

\begin{figure}[h]
    {\includegraphics[width=0.85\textwidth]{img/kdweng.png}}
    \centering
    \caption{Illustration of knowledge distillation mechanism. Figure reprinted from \cite{weng2020acquiring}.}
    \label{img:kdweng}
\end{figure}

\cite{yang2020towards} try to combat issues such as requiring a long time to train NMT models and catastrophic forgetting problem from updating too many of the pre-trained models in the fine-tuning process. They propose a concerted training approach ($CT_{NMT}$) that consists of three different techniques: 1) Asymptotic distillation; 2) Dynamic switch for knowledge fusion; 3) Rate-scheduled updating. The first technique is used to keep the pre-trained knowledge intact. They achieve this by using the pre-trained BERT as a teacher network and the encoder of the NMT models as the student. The goal of the first technique is to mimic the representation coming from the pre-trained BERT by minimizing the cross-entropy loss. The second technique is introduced to perform a combination between the representation from BERT and the encoder of NMT. They achieve this by using a gating mechanism that uses the source as the input to decide how to fuse the representations. The intuition of this technique is that for a particular sentence, BERT might provide a better representation than the currently trained model. Using this gating mechanism, they try to adjust the use of BERT representation in the encoder dynamically. The intuition of this gating mechanism is that BERT might better encode some sentences but not all. The last technique uses a scheduling policy for adjusting the learning rate. They propose using this as they found that updating BERT LM and the NMT at different paces would benefit the final model. An illustration of the asymptotic distillation and dynamic switch mechanism is depicted in Figure \ref{img:ctnmt}.

\begin{figure}[h]
    {\includegraphics[width=0.95\textwidth]{img/ctnmt.png}}
    \centering
    \caption{Illustration of asymptotic distillation and dynamic switch. Figure reprinted from \cite{yang2020towards}.}
    \label{img:ctnmt}
\end{figure}

\cite{chen2019distilling} notices the discrepancy between the objective used to pre-train BERT and the objective used in common NLG tasks such as MT. They propose to introduce knowledge distillation approach learned in BERT for text generation tasks. The first proposal introduces a new objective called Conditional Masked Language Modeling (C-MLM). This objective is inspired by MLM but induces another conditional constraint while fine-tuning the pre-trained BERT on a target dataset. The second approach consist of leveraging the fine-tuned BERT as a teacher model and using the logits output as the learning target for the student network to mimic. They claim that the proposed approach improves the generation output as they now leveraging BERT's bidirectional ability to \textit{plan ahead}. They also claim that BERT's ability to look forward into the future can act as an effective regularizer that can help to boost the quality of the generated output.

\begin{figure}[h]
    {\includegraphics[width=0.95\textwidth]{img/bert_distill.png}}
    \centering
    \caption{Illustration of seq2seq architecture. Figure reprinted from \cite{chen2019distilling}.}
    \label{img:bert_distill}
\end{figure}

\section{Adapters in Sequence-to-Sequence}
\label{sec:adapter_seq}

In this section, we divide the discussions and reviews into three areas. \cref{sec:adapter_place} discusses the importance of adapters placement in the transformer layers as well as in the sequence to sequence framework. The first part reviews the work of \cite{houlsby2019parameter} and \cite{bapna2019simple}. Both works propose the same concept of adapters with similar architecture. The difference between them lies in the placement of the adapters within the transformer layer. The other work by \cite{winata2020adapt} discuss the importance of adapters in sequence-to-sequence framework. They found interesting results where adapters in encoder have more impact than in the decoder.

\cref{sec:app_nlu_asr} and \cref{sec:app_mt} review the application of adapters in other fields such as NLU and ASR as well as in the machine translation. Each will discuss the type of adapters used, the purpose of the adapters, and the impact of including the adapters as a part of their training procedure.

\subsection{Placement of Adapters}
\label{sec:adapter_place}
There are two types of adapters placement in the transformer architecture. The first type of adapter was proposed by \cite{houlsby2019parameter}. We recall that each transformer layer consisted of two sub-layers: a self-attention layer and a feedforward layer. Each sub-layers will be followed immediately by a projection layer whose job is to transform the feature's size back to the original input size. The adapters are applied to the output of the sub-layer of the transformer. Specifically, it is applied to the output of the projection layer that transforms the vectors back to the input size and before being applied to the skip connection. The illustration of the adapter architecture can be seen in Figure \ref{img:ada_houlsby}.

\begin{figure}[h]
    {\includegraphics[width=0.95\textwidth]{img/adapter_houlsby.png}}
    \centering
    \caption{Illustration of adapter architecture. Figure reprinted from \cite{houlsby2019parameter}.}
    \label{img:ada_houlsby}
\end{figure}

The second work of adapters is published by \cite{bapna2019simple}. They take a simpler approach in contrast to \cite{houlsby2019parameter} in regards to incorporating the adapters into the transformer network. Rather than inserting two serial adapters into the sub-layers of the transformer, they instantiate a single instance of adapters and place it on the top of each layer. In addition to the simpler design of adapter layers, they add layer normalization to normalize the input of the adapters. The reasoning behind this is to make the adapters pluggable into any part of the base networks, ignoring the distribution variations of the previous layers. For illustration we refer to Figure \ref{img:ada_bapna}.

\begin{figure}[h]
    {\includegraphics[width=0.95\textwidth]{img/adapter_bapna.png}}
    \centering
    \caption{Illustration of adapter architecture. Figure reprinted from \cite{bapna2019simple}.}
    \label{img:ada_bapna}
\end{figure}

\cite{pfeiffer2021adapterfusion} mentioned that the placement of adapter parameters within a pre-trained model is non-trivial and thus requires extensive experiments. To identify the best setup for the placement of the adapter, the authors perform an exhaustive search on the hyperparameters. This includes the position and the number of adapters within a single layer, the position of residual connections, the bottleneck residual factors, as well as the non-linearity within the bottleneck adapter layer. In their experiments that involve both NLU and NLG, they find the best performing adapter is close to the simple architecture proposed by \cite{bapna2019simple} and not from \cite{houlsby2019parameter}.

We now discuss the placement of the adapters from the perspective of the encoder-decoder framework (sequence-to-sequence). \cite{winata2020adapt} conducted an experiment using similar adapter architecture and placement to \cite{bapna2019simple} in ASR. The encoder in this work represents a model whose job is to encode audio representation. The decoder is responsible for encoding the text features and generating text output. In this work, the authors perform experiments comparing the effectiveness of adapters in both encoder and decoder. They initialized the decoder using pre-trained mBERT weights \cite{devlin2018bert}. Based on their findings, adapters in the decoder are not as effective as they are in the encoder component. These results indicate that adapting the weights in the audio space is more effective for the model's performance than in the text space. Features in mBERT may have already provided enough good features so that further fine-tuning may no longer be necessary.

\subsection{Adapters in NLU and Automatic Speech Recognition}
\label{sec:app_nlu_asr}
Adapters have been used as an alternative for naive fine-tuning in various fields in NLP. \cite{houlsby2019parameter} introduces adapters to fine-tune BERT model in various NLU tasks. Primarily, they use the adapters in GLUE \cite{wang2018glue} and additional classification tasks. They conducted another experiment on a more complicated problem such as SQuAD \cite{rajpurkar2018know}. They found positive results in using adapters in GLUE, other classification, and SQuAD tasks compared to naive fine-tuning. By adding a small number of parameters, they achieved a comparable performance in all tasks compared to fine-tuning the whole weights. We can refer to \cref{sec:adapter_place} for the explanation and illustration of the adapters architecture of their work.

\cite{pfeiffer2020madx} experiments with adapters involve bootstrapping pre-trained NLP models such as BERT, and XLM \cite{conneau2019cross} in low-resource languages. They perform two types of adaptation, language-adaptation and task-adaptation. For this purpose, they use two different adapters in different situations. The language-adapters are trained with the MLM objective to capture various features in different languages. They then put task adapters on top of the language adapters for further fine-tuning for each of the tasks in their experiments. They use an efficient adapter architecture based on \cite{pfeiffer2021adapterfusion} which has similarities to the work of \cite{bapna2019simple}. For more details on the architecture definition, we refer the reader to \cref{sec:adapter_place}. They perform their experiments on three different tasks: Named Entity Recognition (NER), extractive question answering (QA), and causal commonsense reasoning (CCR). These tasks are available in multilingula test sets that cover both high-resource and low-resource languages. We refer the reader to their original paper for more details on the dataset they used on each task. Their main finding is that the task-specific adapters perform similarly to the work of \cite{houlsby2019parameter}. However, since their main focus is on multilingual setup, they did not find satisfying results due to low output quality in unseen languages. With the help of language adapters, they have improved the performance across all tasks. The performance is especially appealing in low-resource languages.

\cite{winata2020adapt} evaluate adapters performance in multilingual ASR setup. They use a similar setup as \cite{bapna2019simple} to adapt both the encoder and decoder. They employ adapters to combat language mismatch and improve models' robustness in various languages with limited resources. Their main finding for the adapters is that it improves the performance across various languages. Furthermore, they also performed experiments to generate a cluster of languages where similar languages are put within the same group. They then share the adapters to languages belonging to the same group and find out that this helps performance in low-resource languages but has small drawbacks in the high resource languages.

\cite{lee2021adaptable} propose multi-domain adapters for language model in ASR setup. Specifically, they employ the adapters in the language model (LM) component for rescoring purposes in an ASR system. They use the similar architecture of the adapter to the work of \cite{houlsby2019parameter} where they introduce adapters within the sub-networks as well as at the top of the layer. Their experiments show that by using separate adapters for each domain, they can re-use a general domain LM and switch domains by replacing adapters with the one that has already been fine-tuned on the respective domain. They also found that with the help of adapters, they manage to improve the performance on a specific problem, such as predicting proper nouns.

\subsection{Adapters in Machine Translation}
\label{sec:app_mt}
\cite{bapna2019simple} propose an alternative to \cite{houlsby2019parameter}. Instead of using two different adapters within a single transformer layer, they propose to use a single adapter on top of the layer. Another difference is that they add layer normalization after the transformer layer output. Furthermore, they also experiment with the adapters in the NLG domain instead of NLU. They use machine translation objective for the pre-training objective and treat the adapters as the adaptation module for a specific language pair. They first pre-trained the model in a large parallel corpus such as WMT before performing domain adaptation on a smaller corpus such as IWSLT in the same language pair. In a single language pair experiment, they found out that the architecture of the proposed adapter is flexible with respect to the size of the data when the adapter's capacity is properly adjusted to match the requirements for the corpus size adaptation. Furthermore, they found out that they did not find any sign of overfitting; instead, they found the model to reach its peak and stay steady throughout the training process.

\cite{philip2020monolingual} apply adapters in multilingual machine translation differently than \cite{bapna2019simple}. \cite{bapna2019simple} use adapters for every target pair. For example, in the English French pair, they have to create two different adapters for English$\rightarrow$French and French$\rightarrow$English. \cite{philip2020monolingual} propose to use a monolingual adapter where instead of using a single adapter for a language pair, they use a single adapter for each of the languages. They use the same adapter architecture as in \cite{bapna2019simple}. Their findings suggest that their approach reduced the number of adapters from $n(n-1)$ to $2n$. Furthermore, they also find that this approach has better performance in low-resource languages than the one proposed in \cite{bapna2019simple}.

\cite{guo2021adaptive} show that adding adapters to BERT during the fine-tuning can be beneficial in the machine translation task. The purpose of the adapters in this work is to make the fine-tuning process more lightweight than the regular fine-tuning. They follow the adapter architecture and use a similar mechanism to include the adapters on the base model as \cite{bapna2019simple}. The difference is that \cite{bapna2019simple} do not use BERT as the base model and use a plain transformer that is pre-trained using machine translation objective.

% The difficulties of incorporating BERT in machine translation has already explained in \ref{sec:domain_adapt}. \cite{guo2021adaptive} also propose that


% \section{Fine-tuning in Sequence-to-Sequence Framework}
% Target 5 pages

\chapter{Experiment Setup}
In this chapter we describe our selection of dataset, framework, and automatic evaluation we used for the experiments. We start by describing a set of dataset as well as the tokenization that we use in Section \ref{sec:dataset} as well as our reasoning on choosing the dataset. We then move forward to the framework we use to implement the neural network, training, and evaluation phase in Section \ref{sec:framework}. Finally, we will discuss automatic evaluation we use during the experiments in Section \ref{sec:aeval}.

\section{German-to-English Dataset}
\label{sec:dataset}
The scope of our experiment is in a single language pair German$\rightarrow$English. We only select a single language pair as we want to focus our experiment on understanding the behaviour of BERT and the adapters in machine translation domain and not focusing on generalization in multiple language pairs. We select IWSLT14 and WMT19 as our primary dataset. IWSLT14 will be mainly used as the dataset for fine-tuning and testing the final performance of the model, while WMT19 is used for the additional dataset in pre-training as well as normal training in some of our baselines.

\subsection{IWSLT}
The 2014 IWSLT evaluation \cite{Cettolo2014ReportOT} continued along the line set in 2010, by focusing on the translation of TED Talks, a collection of public speeches covering many different topics. All TED talks have English captions, which then translated into various languages by volunteers around the world. Translating TED Talks implies dealing with spoken rather than written language, which is hence expected to be structurally less complex, formal and fluent. Moreover, as human translations of the talks are required to follow the structure and rythm of the English captions, a lower amount of rephrasing and reordering is expected than in ordinary translation of written documents.

From an application perspective, TED Talks suggest translation tasks ranging from off-line translation of written captions, up to on-line speech translation, requiring a tight integration of MT with ASR possibly handling stream-based processing.

For each official and optional translation direction, in-domain training and development data were supplied through the website of WIT3 [11], while out-of-domain training data through the workshop's website. As usual, some of the talks added to the TED repository during the last year have been used to define the new evaluation sets (tst2014), while the remaining new talks have been included in the training sets. For reliably assessing progress of MT systems over the years, the evaluation sets tst2013 of edition 2013 were distributed together with tst2014 as progressive test sets,. Development sets (dev2010, tst2010, tst2011 and tst2012) are either the same of past editions or, in case of new language pairs, have been built upon the same talks.

Evaluation sets tst2014 of DeEn derive from those prepared for ASR/SLT tracks, which consist of TEDx talks delivered in German and Italian language, respectively; therefore, no overlap exists with any other TED talk involved in other tasks. Since the De$\rightarrow$En TEDx based MT task was proposed in 2013 as well, the tst2013 has been released as progressive test set. A single TEDx based development set was released for each pair, together with standard TED based development sets dev2010, tst2010, tst2011 and tst2012 sets. The full statistics of the dataset can be seen on \ref{tab:iwslt14stat}.

\begin{table}[h]
    \centering
    \begin{tabular}{@{}cclll@{}}
        \toprule
        \multicolumn{2}{c}{\multirow{2}{*}{\textbf{set}}} & \multicolumn{1}{c}{\multirow{2}{*}{\textbf{sent}}} & \multicolumn{2}{c}{\textbf{tokens}}                                           \\ %\cmidrule(l){4-5}
        \multicolumn{2}{c}{}                              & \multicolumn{1}{c}{}                               & \multicolumn{1}{c}{\textbf{En}}     & \multicolumn{1}{c}{\textbf{De}}         \\ \toprule
        \multicolumn{2}{c}{train}                         & 172k                                               & 3.46M                               & 3.24M                                   \\ \midrule
        \multirow{4}{*}{dev}                              & TED.dev2010                                        & 887                                 & 20,1k                           & 19,1k \\
                                                          & TED.tst2010                                        & 1,565                               & 32,0k                           & 30,3k \\
                                                          & TED.tst2011                                        & 1,433                               & 26,9k                           & 26,3k \\
                                                          & TED.tst2012                                        & 1,700                               & 30,7k                           & 29,2k \\ \midrule
        \multirow{5}{*}{test}                             & TED.tst2013                                        & 993                                 & 20,9k                           & 19,7k \\
                                                          & TED.tst2014                                        & 1,305                               & 24,8k                           & 23,8k \\
                                                          & TEDx.dev2012                                       & 1,165                               & 21,6k                           & 20,8k \\
                                                          & TEDx.tst2013                                       & 1,363                               & 23,3k                           & 22,4k \\
                                                          & TEDx.tst2014                                       & 1,414                               & 28,1k                           & 27,6k \\ \bottomrule
    \end{tabular}
    \caption{Statistics of IWSLT 2014 German$\rightarrow$English dataset.}
    \label{tab:iwslt14stat}
\end{table}

\subsection{WMT}
The Fourth Conference on Machine Translation (WMT) held at ACL 2019 \cite{barrault-etal-2019-findings} hosts a number of shared tasks on various aspects of machine translation. This conference builds on 13 previous editions of WMT as workshops and conferences (\cite{koehn-monz-2006-manual}; \cite{callison-burch-etal-2007-meta}, \cite{callison-burch-etal-2008-meta}, \cite{callison-burch-etal-2009-findings}, \cite{callison-burch-etal-2010-findings}, \cite{callison-burch-etal-2011-findings}, \cite{callison-burch-etal-2012-findings}; \cite{bojar-etal-2013-findings}, \cite{bojar-etal-2014-findings}, \cite{bojar-etal-2015-findings}, \cite{bojar-etal-2016-findings}, \cite{bojar-etal-2017-findings}, \cite{bojar-etal-2018-findings}).

The primary objectives of WMT are to evaluate the state of the art in machine translation, to disseminate common test sets and public training data with published performance numbers, and to refine evaluation and estimation methodologies for machine translation.

The dataset was collected from various sources in the internet. As we have mentioned before, we use WMT for pre-training dataset and an additional dataset to train our baseline models. For this reason, we are not utilizing the dev and test set. Therefore, we show the statistics of the dataset in Table xxx only for the training set. We can see from the Table that the dataset comprises from various news sources. Apart from the large number of sentences, another reason of choosing WMT19 as our additional dataset as it contains sentence pairs from various domains.

\begin{table}[h]
    \centering
    \begin{tabular}{@{}clll@{}}
        \toprule
        \multirow{2}{*}{\textbf{corpus}} & \multicolumn{1}{c}{\multirow{2}{*}{\textbf{sent}}} & \multicolumn{2}{c}{\textbf{tokens}}                                   \\
                                         & \multicolumn{1}{c}{}                               & \multicolumn{1}{c}{\textbf{De}}     & \multicolumn{1}{c}{\textbf{En}} \\ \midrule
        Europarl Parallel Corpus         & 1,825,741                                          & 48,125,049                          & 50,506,042                      \\
        News Commentary Parallel Corpus  & 329,506                                            & 8,363,213                           & 8,295,418                       \\
        Common Crawl Parallel Corpus     & 2,399,123                                          & 54,575,405                          & 58,870,638                      \\
        ParaCrawl Parallel Corpus        & 31,358,551                                         & 559,348,288                         & 598,362,329                     \\
        EU Press Release Parallel Corpus & 1,480,789                                          & 29,458,773                          & 30,097,541                      \\
        WikiTitles Parallel Corpus       & 1,305,135                                          & 2,817,660                           & 3,271,223                       \\ \bottomrule
    \end{tabular}
    \caption{Statistics of WMT 2019 German$\rightarrow$English dataset.}
    \label{tab:wmt19stat}
\end{table}

\subsection{Segmentation}
% - Using huggingface implementation of WordPiece https://huggingface.co/docs/transformers/tokenizer_summary
BERT uses a subword tokenization algorithm WordPiece \cite{schuster2012japanese} to construct the list of vocabularies. The algorithm is very similar to Byte-Pair Encoding (BPE) \cite{sennrich-etal-2016-neural}. BPE works by relying on a pre-tokenizer to split words within the training data such as simple whitespace tokenization.

After the pre-tokenization and a set of unique words as well their frequency has been calculated and gathered, BPE starts by building a symbol vocabulary that consists of all symbols within the corpus. The symbol can consist of anything from alphabet, numeric, and other symbols. BPE then learns a set of rules to merge and form a new symbol from two other symbols from the existing vocabulary. This process is repeated until the number of vocabulary matches the desired number of vocabulary that has already determined. The number of vocabulary is the hyperparameter for BPE.

To provide better example, let's assume we have the following words and their frequency after pre-tokenization\footnote{Example is taken from \url{https://huggingface.co/docs/transformers/tokenizer_summary}}:

\bigskip
``("hug", 10), ("pug", 5), ("pun", 12), ("bun", 4), ("hugs", 5)''
\bigskip

From these words, we then have a set of unique symbols: ``["b", "u", "n", "p", "h", "g", "s"]''. BPE then starts the merging process by using the total frequency of each possible symbol pair. The pair that occurs the most will be pick as a new vocabulary. In our example, we have "h" followed by "u" with a total of 15 times (10 times in hug and 5 times in hugs) and "u" followed by "g" with a total of 20 times. Therefore, we pick "ug" and append the new symbol to the list of vocabulary. We repeat this process until we meet the desired total number of vocabularies.

During the decoding process and assumming now we have the following set of unique symbols: ``["b", "u", "n", "p", "h", "g", "s", "ug"]'', the tokenization will perform by matching the sub-word of the incoming word to the existing vocabulary. For example, if the incoming word is "mug", we will have "<unk> ug" as our tokenization. "<unk>" is introduced to handle tokens/symbols that do not exist in the vocabulary. On the other hand, for word "bug", it will be tokenized into "b ug".


\section{Framework}
\label{sec:framework}
\subsection{HuggingFace Transformer}
Transformers is a library dedicated to supporting Transformer-based architectures and facilitating the distribution of pretrained models. At the core of the libary is an implementation of the Transformer which is designed for both research and production. The philosophy is to support industrial-strength implementations of popular model variants that are easy to read, extend, and deploy. On this foundation, the library supports the distribution and usage of a wide-variety of pretrained models in a centralized model hub. This hub supports users to compare different models with the same minimal API and to experiment with shared models on a variety of different tasks.

Continuously maintained by Huggingface team and contributed by over 400 external contributors outside of Huggingface is one of our reasons in choosing Huggingface to conduct the experiment in this work. The library is released under the Apache 2.0 license and is freely available to download on GitHub\footnote{\url{https://github.com/huggingface/}} and their official website\footnote{\url{https://huggingface.co}}. Furthermore, the website also provides easy to understand tutorials and detailed documentation of the API.

\subsection{AdapterHub}
Another reason why we choose Huggingface is the availability of AdapterHub \cite{pfeiffer-etal-2020-adapterhub}. Despite adapter's simplicity and achieving strong results in multi-task and cross-lingual transfer learning \cite{pfeiffer2021adapterfusion,pfeiffer2020madx}, reusing and sharing adapters was not yet straightforward. The reason is because adapters are rarely released independently due to their subtle difference in architecture as well as strong dependence on the base model, task, and language. To mitigate these issues, AdapterHub is created to facilitate the easiness of training models with adapters as well as sharing the fine-tuned adapters in various settings.


\section{Automatic Evaluation}
\label{sec:aeval}
Given the problems of manual evaluation methods described above, it is natural to try to find a fast, cheap, deterministic and replicable metric. Moreover, it would be a plus if the metric allowed automatic model optimization.

With these properties, the proposed metric can be used to check progress, allow researchers to iterate and evaluate their proposals faster and speed up the development of the field.

The \textit{BLEU} metric (Bilingual Evaluation Understudy, \cite{BLEU}) is one of these automatic evaluation metrics, which is widely used in the field of MT.
It evaluates an output (sentence or corpus) of an MT system (the candidate) by comparing it with correct translations (the references).

The two main components of BLEU are the n-grams precisions and length of the candidate.
Precision is very commonly used in the machine learning field.
In the case of BLEU, it measures the percentage of correct n-grams in the candidate.
The trivial case is unigram ($n=1$) precision which is merely the ratio of the number of tokens shared between candidate and reference divided by the number of tokens in the candidate.
However, this simple definition of precision would not be very precise in some cases, for example:

\bigskip

\textbf{Candidate}: \underline{that} \underline{that} that

\textbf{Reference}: I think \underline{that} it is not \underline{that} bad

\bigskip

The straightforward (lowercase) unigram precision of the above example is 1.0 (100\%), even though only two \textit{that} unigrams in the candidate are matched with the two unigrams in the reference.
That is to say, the number of n-grams shared between the candidate and the reference should be clipped to the number of n-grams that appear in the reference.
After that modification, the \textit{modified n-gram precision} in BLEU is computed as follows:

\begin{equation}
    p_n=\frac{\sum_{C\in\{Candidates\}}\sum_{n-gram\in C}Count_{clip}(n-gram)}{\sum_{C'\in\{Candidates\}}\sum_{n-gram'\in C'}Count(n-gram')}
\end{equation}

The second problem BLEU has to deal with is erroneously short candidates.
Take the following example:

\bigskip

\textbf{Candidate}: that

\textbf{Reference}: I think \underline{that} it is not \underline{that} bad

\bigskip

Although the candidate definitely does not express enough information compared to the reference, the precision of this case is $1.0$.
To penalize such output from MT systems, BLEU introduced the \textit{brevity penalty} where $c$ and $r$ are the length of the candidate and the length of the reference, respectively.

\begin{equation}
    BP=\begin{cases} 1 & \mbox{if } c>r \\ e^{(1-r/c)} & \mbox{if } c\le r \end{cases}
\end{equation}

When there are more than one reference, $r$ is called the \textit{effective reference length} and it is taken as the length of the reference that is closest to the length of the candidate.
It is important to note that which reference is the closest varies between implementations of BLEU, see the example below. Both references' lengths are one token different away from the candidate.

\bigskip

\textbf{Candidate}: I like

\textbf{Reference} 1: I like it

\textbf{Reference} 2: I

\bigskip

We advise the reader to use the official BLEU evaluation script used by the Workshop of Machine Translation (WMT) shared task,\footnote{\url{ftp://jaguar.ncsl.nist.gov/mt/resources/mteval-v13a.pl}} or its Python reimplementation.\footnote{\url{https://github.com/mjpost/sacreBLEU}}

Combining those two main components, the BLEU score is defined as follows:

\begin{equation}
    BLEU=BP\cdot exp\left( \sum_{n=1}^{N} w_n \log p_n \right)
\end{equation}

Specifically, BLEU computes the n-grams precisions $p_n$ of the given candidate and references (by default from unigrams to 4-grams).
It then geometrically averages them with predefined weights $w_n$ (all set to $1/4$ by default), and scales down the score in the case of inadequately short candidates with the brevity penalty.

% Target: 35 pages
% Current: 3

\chapter{Adapters in Machine Translation}
\label{chap:adaptmt}
This chapter aims to study the impact of pre-trained models when fine-tuning machine translation models with adapters. Specifically, we are interested in understanding the contribution of good representation in the pre-trained language model to the adapters during fine-tuning. Furthermore, we are also interested in understanding the capability of adapters to perform with artificially degraded pre-trained models. We propose to use transformer-based architecture such as BERT (pre-trained or trained from scratch) for both the encoder and decoder components while using adapters for fine-tuning. To be more specific, we divide the experiments into four different pre-trained model setups:
\begin{itemize}
    \item Use BERT weights\footnote{We use publicly available BERT weights from Huggingface hub \url{https://huggingface.co}} as the pre-trained weights (\texttt{Pre-trained BERT}). We use this model as the baseline for our experiments in this chapter.
    \item Use randomly initialised BERT models and pre-train the models with MLM objective on IWSLT and WMT data (\texttt{Pre-trained Transformer}). This setup is used to understand the impact of different data volumes and domains on the quality of the pre-trained models that will be fine-tuned with adapters in the later phase.
    \item Use the pre-trained BERT model where the weights within the same layer are shuffled (\texttt{Pre-trained shuffled}). We use this setup to understand whether the adapters can capture important features from the original BERT weights and restore the performance even though the weights are no longer in the original position.
    \item Use randomly initialised BERT models with no pre-training (\texttt{Pre-trained random}). We use this setup as a complement of \texttt{Pre-trained shuffled} experiments to understand the impact of pre-trained models that contain relatively inferior knowledge than the original BERT model on the adapters when fine-tuned in the MT task.
\end{itemize}

\section{Experiments Setup}
\subsection{Language Model}
\label{ssec:langmodel}
\subsubsection{Dataset}
From \cite{devlin2018bert}, we understand that BERT was trained with billions of words from various sources and domains. However, we do not fully understand the proper condition to stop adding more sentences to the pre-training data so that we can reduce the hours of training the model. Additionally, we do not know the impact of combining different domains in the pre-training data on the final performance of the model. For those reasons, we are reducing the scope of the pre-training data by restricting the creation of pre-training data only from machine translation datasets in two different domains and constructing the pre-trained models with different sizes of data. We use WMT and IWSLT for the experiment as WMT and IWSLT contain different domains, and WMT has a significantly larger volume and longer sentences. Specifically, we construct three different datasets with different volumes:
\begin{enumerate}
    \item A standalone IWSLT.
    \item A combination of IWSLT and WMT data with a total of 500k sentences. We add the WMT data on top of IWSLT to increase the volume of our pre-training data. We should also note that the domain of WMT differs from the IWSLT, and we only test the models on the IWSLT test set. The WMT data can thus lead to worse translation performance due to domain mismatch.
    \item A combination of IWSLT and WMT data with 2 million sentences. The same as (2) but with larger volumes.
\end{enumerate}

Datasets (2) and (3) were constructed with a simple approach by using all training data from IWSLT, randomly selecting sentences from the WMT dataset, and combining them until we met the required number of sentences mentioned in the previous paragraph.

\begin{table}[]
    \centering
    \begin{tabular}{@{}|l|l|l|l|@{}}
        \toprule
        \multicolumn{1}{|c|}{\textbf{Dataset}}                                                          &
        \multicolumn{1}{c|}{\textbf{Initial Weights}}                                                   &
        \multicolumn{1}{c|}{\textbf{\begin{tabular}[c]{@{}c@{}}Used for \\ pre-training?\end{tabular}}} &
        \multicolumn{1}{c|}{\textbf{\begin{tabular}[c]{@{}c@{}}Used for \\ fine-tuning?\end{tabular}}}                                 \\ \midrule
        \multirow{2}{*}{Standalone IWSLT}                                                               & Pre-trained BERT & No  & Yes \\ \cmidrule(l){2-4}
                                                                                                        & Random BERT      & Yes & Yes \\ \midrule
        IWSLT + WMT 500k                                                                                & Random BERT      & Yes & No  \\ \midrule
        IWSLT + WMT 2m                                                                                  & Random BERT      & Yes & No  \\ \bottomrule
    \end{tabular}
    \caption[IWSLT and WMT data usages with respect to the weights initialization]{The initial weights represent the weights used on both the encoder and decoder. Pre-trained BERT refers to the pre-trained models that use BERT. Random BERT models use BERT configuration but not the weights. Therefore, contrary to the Pre-trained BERT, Random BERT models undergo a pre-training process before fine-tuning. The pre-trained and fine-tuned columns represent the boolean value to mark whether the models initialized with the targeted weights are pre-trained or fine-tuned with the respective data. For example, Huggingface BERT weights are not pre-trained with IWSLT but use IWSLT for fine-tuning.}
    \label{tab:mixeddataset}
\end{table}

To illustrate how the datasets are used in the experiments, we refer to \cref{tab:mixeddataset}. We can see that datasets other than standalone IWSLT are used only for pre-training. The IWSLT is always used for fine-tuning, and depending on the initial weights, we also use it as pre-training.

\subsubsection{Model}
To pre-train the model, we follow the work of \cite{devlin2018bert} by using the Masked Language Model (MLM) objective. In every sentence, some words will be masked, and the model has to predict the original words. A complete illustration can be found in \cref{img:mlmobj}.

\begin{figure}[h]
    {\includegraphics[width=0.85\textwidth]{img/mlm_obj.png}}
    \centering
    \caption{Illustration of the MLM objective during the pre-training. The illustration is reprinted from \cite{Park2019SelfSupervisedCD}.}
    \label{img:mlmobj}
\end{figure}

The MLM is introduced as an alternative objective to the auto-regressive language model. The auto-regressive objective is inherently different from the MLM. In auto-regressive, the models have to predict the word at step $t$, and they only have the ability to look at the previous words ($t-1...0$) as the context. Auto-regressive is used in a couple of works such as GPT models \citewithpar{brown2020language,ratford2019language,radford2018improving} for pre-training. One of the advantages of using the MLM compared to the auto-regressive is that we can exploit the bidirectional context rather than only predicting the words from left to right.

We use the default BERT configuration from Huggingface to construct the model. The complete list of hyperparameters can be found in \cref{tab:hyp}.

\begin{table}[]
    \begin{tabular}{@{}lccc@{}}
        \toprule
        \textbf{Name}                   & \textbf{En} & \textbf{De} & \textbf{Description}                                                                         \\ \midrule
        vocab\_size                     & 30522       & 31102       & \begin{tabular}[c]{@{}c@{}}Vocabulary size of the\\ BERT model\end{tabular}                  \\ \midrule
        hidden\_size                    & 768         & 768         & \begin{tabular}[c]{@{}c@{}}Dimensionality for each\\ of the layers\end{tabular}              \\ \midrule
        num\_hidden\_layers             & 12          & 12          & \begin{tabular}[c]{@{}c@{}}Number of hidden layers\\ in the transformer\end{tabular}         \\ \midrule
        num\_attention\_heads           & 12          & 12          & \begin{tabular}[c]{@{}c@{}}Number of attention heads for\\ each attention layer\end{tabular} \\ \midrule
        intermediate\_size              & 3072        & 3072        & \begin{tabular}[c]{@{}c@{}}Dimensionality of the\\ feed-forward layer\end{tabular}           \\ \midrule
        hidden\_act                     & gelu        & gelu        & \begin{tabular}[c]{@{}c@{}}The activation function\\ within the layer\end{tabular}           \\ \midrule
        learning\_rate                  & 0.0005      & 0.0005      & \begin{tabular}[c]{@{}c@{}}Learning rate used\\ for the experiments\end{tabular}             \\ \midrule
        optimizer                       & adam        & adam        & \begin{tabular}[c]{@{}c@{}}Model optimizer\end{tabular}                                      \\ \midrule
        batch\_size                     & 64          & 64          & \begin{tabular}[c]{@{}c@{}}Batch size used\\ for the experiments\end{tabular}                \\ \midrule
        hidden\_dropout\_prob           &
        0.1                             &
        0.1                             &
        \begin{tabular}[c]{@{}c@{}}The dropout probability\\ for all fully connected layers in\\ the embeddings, encoder,\\ and pooler\end{tabular}                \\ \midrule
        attention\_probs\_dropout\_prob &
        0.1                             &
        0.1                             &
        \begin{tabular}[c]{@{}c@{}}The dropout ratio for the\\ attention probabilities\end{tabular}                                                                \\ \bottomrule
    \end{tabular}
    \caption{Transformer BERT hyperparameters for English (based on \texttt{bert-base-uncased}) and German (based on \texttt{bert-base-german-dbmdz-uncased}).}
    \label{tab:hyp}
\end{table}

We train the model until convergence with a maximum of 500 epochs. We define convergence by training for at least one day, with the validation loss no longer improving. We train the models in the MetaCentrum cluster with a specific configuration where the running jobs will be automatically killed after 24 hours. The loss usually starts to converge in less than 24 hours, but in some cases, we still see some models that are not yet converged. In this case, we continue the training by running another job process and loading the last checkpoint. On the other hand, if the loss starts to diverge before 24 hours or 500 epochs, we stop the training manually and select the checkpoint with the lowest loss score. Each language may end up converging in a different number of steps.

\subsection{Machine Translation}
\subsubsection{Dataset}
In machine translation experiments, there are several scenarios in which we use both WMT and IWSLT datasets. The IWSLT dataset is mainly used to fine-tune and evaluate the final model. The WMT, on the other hand, will be combined with IWSLT for training the baseline models. We use three different datasets for the experiments similar to the language model setup: IWSLT standalone, IWSLT + WMT (500k), IWSLT + WMT (2 million). To be more specific, we use the IWSLT standalone dataset for training, evaluation, and testing. The remaining data setups are used only for training, and we limit ourselves to the IWSLT data for evaluation and testing.

\subsubsection{Model}
We use the seq2seq architecture by \cite{vaswani2017attention}. The seq2seq architecture contains two different components, the encoder and the decoder. This architecture's details and illustration have been described in \cref{ssec:transformer}. The encoder and decoder use the same model and hyperparameters as described in \cref{ssec:langmodel}. The only modification made from the original language model is on the decoder side. On the encoder side, we only use the self-attention layers to gather some context from the neighbours within the same layer. On the other hand, the decoder requires further context to generate its outputs by including the vector representation from the encoder as additional features. For this reason, an extra component named \texttt{cross\_attention} layer is introduced in the decoder, and it is trained from scratch in all experiments.

\section{Experiments Results}
This section discusses the results obtained for machine translation. First, in \cref{ssec:adaptcomp}, we conduct the comparative study of using adapters in different pre-trained scenarios. Second, in \cref{ssec:randshuff}, we perform a study by replacing the BERT model with a different BERT version where the weights are shuffled. We continue the experiments using pre-trained models with completely random weights and no pre-training. Finally, in \cref{ssec:randpre}, we perform experiments comparing the performance of the randomly set weights pre-trained model that is fine-tuned with adapters with the baseline models as well as models that are pre-trained with the combination of WMT and IWSLT.

\subsection{Adapters Comparison}
\label{ssec:adaptcomp}
\subsubsection{Experiment Setup and Motivation}
In this section, we conduct experiments to understand the contribution of adapt\-ers by comparing trained and fine-tuned models with different sizes of datasets. The definition of the dataset is the same as have already explained in the previous section. The dataset is used for two different purposes:
\begin{itemize}
    \item Used to train the BERT-style language model from scratch, separated for source and target languages.
    \item Used for pre-training and later fine-tuning the seq2seq model on IWSLT parallel data.
\end{itemize}

\begin{figure}[t]
    {\includegraphics[width=0.85\textwidth]{img/baseline.png}}
    \centering
    \caption[Comparison between baseline models trained with different size of datasets.]{
        Comparison between baseline models trained with different size of datasets. \texttt{baseline} represents the model trained only using IWSLT; \texttt{baseline\_500k} represents the model trained using IWSLT and WMT with total of 500k sentence pairs; \texttt{baseline\_2m} represents the model trained using IWSLT and WMT with total of 2 million sentence pairs.}
    \label{img:basecomp}
\end{figure}

\subsubsection{Experiment Results}

In this section, we compare the result of the baseline models with the fine-tuned models with adapters. Our first experiment compared the results from an empty BERT model (a transformer model with BERT configuration but no pre-trained BERT weights) without any pre-training and adapters and only trained the model from scratch. We refer to these models as our \texttt{baseline} for the rest of this chapter. From \cref{img:basecomp}, we can see that adding more data to the baseline models does not necessarily improve the performance. We suspect the models require more training time to get the best final performance. There is a clear gap between \texttt{baseline\_2m} and the rest of the baseline models. \texttt{baseline} and \texttt{baseline\_500k} perform really well from the start while \texttt{baseline\_2m} lags behind. We suspect this is due to the effect of including more sentences from different domains. We found that \texttt{baseline\_500k} performs the best as it balances the models to avoid overfitting in the intended domain and still benefits from out-of-domain data. \texttt{baseline\_2m} shows the impact on domain difference where it has relatively lower performance than the other models. However, we must acknowledge that the model has not fully converged in the limited training time (measured by the number of steps) we have and has a chance to improve its performance further. Despite the lower performance in BLEU score, \texttt{baseline\_2m} deserves a manual evaluation of the translation output. We argue that in some cases, lower BLEU may not necessarily reflect a lower quality. We perform the manual evaluation at a later stage in this chapter.

To see the impact of adapters, we compare the result on different sizes of pre-training data used for the base model. The base models are then fine-tuned with the adapters module on the IWSLT data. As we can see from \cref{img:adpcomp}, pre-trained BERT with adapters achieves the best performance from the earlier steps compared to the rest of the models. In contrast to the baseline models, we see the benefit of adding more sentences to the pre-training (see \texttt{adapters\_pt\_2m} vs \texttt{adapters\_pt\_500k}). From \cref{img:basecomp} we see that \texttt{baseline\_500k} outperforms \texttt{baseline\_2m} by approximately 5 BLEU despite larger training data on \texttt{baseline\_2m}. Note that the training data we refer to in the baseline models is the training data for MT. On the other hand, when we add more training data on the pre-training side, we can see that the model trained using 500k data has a lower performance than the one using 2 million data. This tells us that adding larger pre-training monolingual data and then fine-tuning the models with adapters impact the performance positively.

\begin{figure}[h]
    {\includegraphics[width=0.75\textwidth]{img/adapterscomparison.png}}
    \centering
    \caption[The effect of pre-training data size when the adapters are then trained on the same IWSLT data.]{
        The effect of pre-training data size when the adapters are then trained on the same IWSLT data. \texttt{baseline} represents the model trained only using IWSLT; \texttt{adapters\_pt} represents the model pre-trained only using IWSLT; \texttt{adapters\_pt\_500k} represents the model pre-trained using IWSLT and WMT with total of 500k sentence pairs; \texttt{adapters\_pt\_2m} represents the model pre-trained using IWSLT and WMT with total of 2 million sentence pairs; \texttt{adapters\_bert} represents the model that uses BERT weights.}
    \label{img:adpcomp}
\end{figure}

In the model that only uses IWSLT as the pre-training data, we observe performance degradation in the middle of the fine-tuning. We found this is due to the gradient explosion in the cross-attention layer, and we attribute this instability to the small data used in the pre-training. In the later steps, the IWSLT model eventually achieved performance similar to the 500k model.

\subsection{Random and Shuffled Pre-trained Weights}
\label{ssec:randshuff}
\subsubsection{Experiment Setup and Motivation}
We perform two different categories of experiments to study how the adapters benefit from the pre-trained weights.
First, we conduct experiments where we shuffle the BERT weights. We separate the weights initialization into two approaches to perform the experiments:
\begin{itemize}
    \item We shuffled the weights from the column perspective. We shuffled the column-wise order of all the matrices while keeping the row-wise order intact.
    \item We shuffled the weights from both column and row perspectives.
\end{itemize}

Second, we conduct experiments where we set random weights on all base network layers and treat them as the pre-trained model. During the fine-tuning, we only update the adapter weights and keep the random weights intact.

\subsubsection{Experiment Results}
We can see from \cref{img:shfrndcmp} that the performance of the model that uses random pre-trained weights is more stable than the one using shuffled BERT weights. All shuffled BERT models suffer from gradient explosion similar to the IWSLT model we show in the previous section. Furthermore, shuffling the weights on both the row and column sides seems more detrimental than just shuffling on the column side. This may show that there could be some pattern in \texttt{adapters\_shuffled} that the adapters can eventually help to recover during the fine-tuning, while on \texttt{adapters\_shuffled\_both} more patterns are missing and more challenging to recover.

Although the performance of the random model (\texttt{adapters\_random}) is still below the baseline model, it is interesting to see that fine-tuning only the adapters, cross-attention, and output layers is sufficient to achieve a reasonable BLEU score, considering that the pre-trained model does not contain meaningful information relative to the original BERT weights. Furthermore, it is also interesting to see that compared to \texttt{adapters\_shuffled\_both}, we can get better final performance even though we can categorize \texttt{adapters\_shuffled\_both} as a model that uses randomly set weights pre-trained model. We hypothesize that this is probably related to the difference in the distribution of the weights between the models. In \texttt{adapters\_random} we are limited to the distribution of the weights in the initialization algorithm used in BERT, while in \texttt{adapters\_shuffled\_*} the distribution may be different as the weights in pre-trained BERT have already been optimized with the MLM objective.

\begin{figure}[h]
    {\includegraphics[width=0.85\textwidth]{img/randomshuffled.png}}
    \centering
    \caption[Comparison between adapters using shuffled BERT and random weights as the pre-trained models.]{Comparison between adapters using shuffled BERT and random weights as the pre-trained models. \texttt{baseline} represents the model trained only using IWSLT; \texttt{adapters\_random} represents the model pre-trained only using random weights; \texttt{adapters\_shuffled} represents the model pre-trained using column-wise shuffled BERT model; \texttt{adapters\_shuffled\_both} represents the model pre-trained using shuffled BERT model.}
    \label{img:shfrndcmp}
\end{figure}

Since we are relying on the pre-trained model with a random set of weights and with no further training, there is a possibility that our method may only work in a single random seed. To ensure a robust experiment, we repeat the random experiments ten times with ten different random seeds. We can see from the result in \cref{img:rndmseed} that all the random seed performs similarly to one another.

\begin{figure}[h]
    {\includegraphics[width=0.85\textwidth]{img/adapter_random_multiseed.png}}
    \centering
    \caption{Comparison of different random seed for randomly set weights based models.}
    \label{img:rndmseed}
\end{figure}

\subsection{Random Pretrained vs. Out-of-Domain Data}
\label{ssec:randpre}
\subsubsection{Experiment Setup and Motivation}
In this experiment, we use the same setup as in \cref{ssec:adaptcomp,ssec:randshuff}. Specifically, we are interested in investigating the model's performance that uses randomly set weights as the pre-trained model compared to the baseline model. We recall that we can gain a reasonable score when fine-tuning the model with adapters from the previous experiments using randomly set pre-trained weights on the transformer model. In this experiment, we are conducting further study to understand the performance relative to the \texttt{baseline} model, where the model was trained using only IWSLT data and a different baseline (\texttt{baseline\_2m}, where the model was trained using a mix of WMT and IWSLT. This comparative study aims to understand whether we can see the benefits of using adapters versus training the whole transformer model with bigger data sizes.

\subsubsection{Experiment Results}

\begin{figure}[h]
    {\includegraphics[width=0.85\textwidth]{img/random.png}}
    \centering
    \caption[Comparison between pre-trained random weights and baseline model.]{Comparison between pre-trained random weights and baseline model. \texttt{baseline} represents the model trained only using IWSLT; \texttt{baseline\_2m} represents the baseline model trained with a combination of IWSLT and WMT sentence pairs; \texttt{adapters\_random} represents the model pre-trained only using random weights.}
    \label{img:rndbslcmp}
\end{figure}

We analyze the random weights by comparing the result with the best-performing baseline and our pre-trained transformer models. From \cref{img:rndbslcmp}, we can see that the performance of the random model achieves a similar result as the baseline model that uses 2 million training sentence pairs. Recall that the baseline model is the BERT-style transformer model trained from scratch without fine-tuning and adapters. This tells us that training the whole model with bigger data does not necessarily improve the model's performance. It may need further tuning to gain the benefits of bigger data and bigger models because any departure from the domain of the test set can be more harmful than useful. We can see the result of using random weights as a potential alternative for training the model with small data such as IWSLT. While the performance is still far from the baseline (no adapters, full model training on IWSLT, i.e. in-domain data), this result shows that the base model's structure helps the adapter achieve a meaningful performance with a tiny portion of weights trained in the fine-tuning.

\begin{table*}[h]
    \centering
    \begin{tabular}{@{}l@{}}
        \toprule
        \multicolumn{1}{c}{\textbf{Random Weights + Adapters}}                                                                                                                                                                                                                                                                                                            \\ \midrule
        \begin{tabular}[c]{@{}l@{}}\textbf{Input}: wir tanzen im tempel und werden zu gott. \& quot ;\\ \textbf{Reference}: we dance in the temple and become god . \&quot;\\ \textbf{Hypothesis}: we \& apos ; re going to be able to become god. \& quot ;\end{tabular}                                                                                                 \\ \midrule
        \begin{tabular}[c]{@{}l@{}}\textbf{Input}: aber gleichzeitig hatten sie eine klare kenntnis des waldes, \\ die erstaunlich war.\\ \textbf{Reference}: but at the same time they had a perspicacious knowledge of the\\ forest that was astonishing.\\ \textbf{Hypothesis}: but at the same time, they had a clear of the audience\\ who was amazing.\end{tabular} \\ \midrule
        \begin{tabular}[c]{@{}l@{}}\textbf{Input}: es ist so wunderbar. ihr musst es beschutzen. \& quot ;\\ \textbf{Reference}: it is that beautiful . it is yours to protect . \&quot;\\ \textbf{Hypothesis}: it \& apos ; s wonderful. you have to protect it. \& quot ;\end{tabular}                                                                                  \\ \bottomrule
    \end{tabular}
    \caption{Prediction results from randomly set pre-trained model fine-tuned with adapters}
    \label{tab:qtrand}
\end{table*}

We perform a quick check of the output of the model in \cref{tab:qtrand}. From the first two lines, we can see that the model has difficulty capturing complex phrases. In the first row, the model missed \textbf{tanzen im tempel} which means \textbf{dancing in the temple}. For the second row, the model confuses \textbf{knowledge of the forest} and outputs \textbf{audience} instead. Furthermore, the model also does not translate the word \textbf{kenntnis} and makes the translation unclear since the object of the sentence is missing. Despite those mistakes, the model still can capture simple sentences, as shown in the third row. Another observation that we noticed in the generated output is the tokenization of \texttt{\&quot\;} and \texttt{\&amp\;}. Instead of being treated as a single token, the tokenizer treats the token as three different subword tokens. We notice that this is due to the unavailability of the aforementioned token in the pre-trained BERT vocabulary.

\section{Qualitative Comparison}
\begin{sidewaystable*}
    \centering
    % \begin{tabular}{|l|l|l|}
    %     \hline
    %     \multicolumn{1}{|c|}{\textbf{Baseline IWSLT}}                                                                                                                                                                                                                     &
    %     \multicolumn{1}{c|}{\textbf{IWSLT + WMT (total 2m)}}                                                                                                                                                                                                              &
    %     \multicolumn{1}{c|}{\textbf{BERT + Adapters}}                                                                                                                                                                                                                                \\ \hline
    %     \begin{tabular}[c]{@{}l@{}}\textbf{input}: erinnerst du dich an\\ den patienten\\ mit dem gereizten rachen? \\ \textbf{prediction}: do you remember\\ reading to the patients? on the\end{tabular}                                                                &
    %     \begin{tabular}[c]{@{}l@{}}\textbf{input}: erinnerst du dich an\\ den patienten\\ mit dem gereizten rachen? \\ \textbf{prediction}: do you remember\\ the patient with the tingling\\ revenge?\end{tabular}                                                       &
    %     \begin{tabular}[c]{@{}l@{}}\textbf{input}: erinnerst du dich an\\ den patienten\\ mit dem gereizten rachen? \\ \textbf{prediction}: remember the\\ patient with\\ the bruised remorse?\end{tabular}                                                                          \\ \hline
    %     \begin{tabular}[c]{@{}l@{}}\textbf{input}: großartig, sagte ich.\\ legte auf.\\ \textbf{prediction}: great, i said.\\ got up..\end{tabular}                                                                                                                       &
    %     \begin{tabular}[c]{@{}l@{}}\textbf{input}: großartig, sagte ich.\\ legte auf.\\ \textbf{prediction}: great, i said.\\ put on..\end{tabular}                                                                                                                       &
    %     \begin{tabular}[c]{@{}l@{}}\textbf{input}: großartig, sagte ich.\\ legte auf.\\ \textbf{prediction}: great, i said.\\ put it down.\end{tabular}                                                                                                                              \\ \hline
    %     \begin{tabular}[c]{@{}l@{}}\textbf{input}: - - aber in unserer\\ entdeckung der welt, haben\\ wir alle arten unterschiedlicher\\ methoden.\\ \textbf{prediction}: but in our discovery\\ of the world, we \& apos ;\\ ve got all sorts of different\end{tabular}  &
    %     \begin{tabular}[c]{@{}l@{}}\textbf{input}: - - aber in unserer\\ entdeckung der welt, haben\\ wir alle arten unterschiedlicher\\ methoden.\\ \textbf{prediction}: - - but in our discovery\\ of the world, we have\\ all kinds of different methods.\end{tabular} &
    %     \begin{tabular}[c]{@{}l@{}}\textbf{input}: - - aber in unserer\\ entdeckung der welt, haben\\ wir alle arten unterschiedlicher\\ methoden.\\ \textbf{prediction}: but in our discovery\\ of the world, we have\\ all sorts of different ways of doing\\ things.\end{tabular} \\ \hline
    % \end{tabular}
    % \begin{table}[]
    \begin{tabular}{@{}l@{}}
        \toprule
        \multicolumn{1}{c}{\textbf{Sample Translation Output}}                                                                                                                                                                                                                                                                                                                                                                                                                                                                                                                            \\ \midrule
        \begin{tabular}[c]{@{}l@{}}\textbf{Input}: erinnerst du dich an den patienten mit dem gereizten rachen?\\ \textbf{Reference}: do you remember that patient you saw with the sore throat?\\ \textbf{Hypothesis 1}: do you remember reading to the patients? on the\\ \textbf{Hypothesis 2}: do you remember the patient with the tingling revenge?\\ \textbf{Hypothesis 3}: remember the patient with the bruised remorse?\end{tabular}                                                                                                                                            \\ \midrule
        \begin{tabular}[c]{@{}l@{}}\textbf{Input}: großartig, sagte ich. legte auf.\\ \textbf{Reference}: great, i said. got off the phone.\\ \textbf{Hypothesis 1}: great, i said. got up..\\ \textbf{Hypothesis 2}: great, i said. put on..\\ \textbf{Hypothesis 3}: great, i said. put it down.\end{tabular}                                                                                                                                                                                                                                                                           \\ \midrule
        \begin{tabular}[c]{@{}l@{}}\textbf{Input}: - - aber in unserer entdeckung der welt, haben wir alle arten unterschiedlicher methoden.\\ \textbf{Reference}: but in our discovery around the world, we have all kinds of other methods.\\ \textbf{Hypothesis 1}: but in our discovery of the world, we \& apos ; ve got all sorts of different\\ \textbf{Hypothesis 2}: - - but in our discovery of the world, we have all kinds of different methods\\ \textbf{Hypothesis 3}: but in our discovery of the world, we have all sorts of different ways of doing things.\end{tabular} \\ \bottomrule
    \end{tabular}
    % \end{table}
    \caption{Prediction results from \textbf{Hypothesis 1}: Baseline model trained with only IWSLT data; \textbf{Hypothesis 2}: Pre-trained model with adapters where we pre-train the model with IWSLT and WMT with a total of 2 million pre-training data; \textbf{Hypothesis 3}: BERT with adapters.}
    \label{tab:qtvout}
\end{sidewaystable*}
% We perform a manual check to compare the generated results on some of our models.
We perform a manual check on the generated translations of some of our models.
From \cref{tab:qtvout}, we can see that none of the models generated the correct result for the first example. However, BERT + adapters and 2 million pre-trained base models generate the proper context where the result is still discussing \textbf{the patient}. The wrong part is when the model generates an incorrect translation regarding the patient's disease. The second example shows that the BERT + adapters model creates a better quality of translation than the other models. The final example shows that BERT + adapters generate an interesting output where it manages to remove unimportant characters such as ``\texttt{--}'' and produce readable output. People may vary in their opinion on this example as the 2 million pre-trained base model generates a more concise output.

We also notice that the BERT-based models have difficulties generating long sentences. The models always cut the translation short when an unavailable token such as \texttt{\&quot\;} appears in the sentence. To give a more explicit example, \cref{tab:errout} illustrates the significant difference in the outputs when the token \texttt{\&quot\;} is observed in the input. This shows that tokenization plays an essential role in the model where tokens unavailable in the vocabulary list may affect the generated output significantly.

\begin{table}[]
    \begin{tabular}{@{}l|l@{}}
        \toprule
        \multicolumn{1}{c}{\textbf{Input}}                                                                                                                                                          &
        \multicolumn{1}{c}{\textbf{Output}}                                                                                                                                                           \\ \midrule

        \begin{tabular}[c]{@{}l@{}}\& quot ; moneyball \& quot ; erscheint\\ bald und dreht sich um\\ statistiken und um diese zu nutzen\\ ein großartiges baseball team aufzustellen.\end{tabular} &
        \begin{tabular}[c]{@{}l@{}}\& quot ; devilball \& quot ; appears\\ soon, and it \& apos ;\end{tabular}                                                                                        \\ \midrule
        \begin{tabular}[c]{@{}l@{}}moneyball erscheint bald\\ und dreht sich um\\ statistiken und um diese zu\\ nutzen ein großartiges baseball\\ team aufzustellen.\end{tabular}                   &
        \begin{tabular}[c]{@{}l@{}}fatball soon appears and it\\ turns out statistics and\\ to use that to build a\\ great baseball team\end{tabular}                                                 \\ \bottomrule
    \end{tabular}
    \caption[An example of bad model behaviour when the input contains unknown tokens]{An example of bad model behaviour when the input contains unknown tokens like \texttt{\&quot\;} (top row). The translation is not abruptly cut if these symbols are removed (bottom row).}
    \label{tab:errout}
\end{table}
% Target: 35 pages
% Current: 3

\chapter{Adapters Effectiveness in Machine Translation}
\label{chap:adaptefct}
In this chapter, we continue the study from the previous chapter to understand more about the relation between adapters and pre-trained models. Like the previous chapter, we use BERT and its variance, per-our definition below, as the base pre-trained models and fine-tune them with adapters. This study aims to evaluate the combination of adapters and transformer models on machine translation tasks and study adapters' effectiveness by putting them only in the encoder or the decoder. We evaluate the models on the machine translation task. We also experiment with down-scaling the pre-trained model size and try to recover the performance from being comparable to the full-sized model. We conduct the experiments by separating them into three different areas:
\begin{itemize}
    \item Use BERT weights\footnote{We use publicly available BERT weights from Huggingface hub \url{https://huggingface.co}} as the pre-train weights and investigate the importance of adapters in encoder or decoder.
    \item Use BERT weights and investigate their importance compared to random weights in the encoder or decoder while fine-tuning with adapters.
    \item Down-scaling BERT weights by either zeroing out half of BERT's weights (\texttt{zbert}) or completely removing them from the weight matrices, squashing the matrices (\texttt{zsbert}). We use the down-scaling technique to understand whether we can use adapters to recover the performance of the original BERT (without adapters) while using fewer parameters.
\end{itemize}

\section{Fixed Variable Parameters of Experients}
\subsection{Framework}
\begin{table}[]
    \centering
    \begin{tabular}{@{}cc@{}}
        \toprule
        \textbf{Name}            & \textbf{Value}        \\ \midrule
        \textbf{Batch size}      & 64                    \\
        \textbf{Learning rate}   & 0.0005                \\
        \textbf{Vocabulary size} & 31102 (de), 30522(en) \\ \bottomrule
    \end{tabular}
    \caption{Fixed hyperparameters throughout the experiments}
    \label{tab:hyp_invest}
\end{table}

As we mentioned at the beginning of the chapter, we have several scenarios we use to conduct the experiments. Despite various possibilities of different setups, we describe variables that we fixed throughout the experiments. As mentioned in Chapter \ref{chap:03}, we use Huggingface as our main framework with added modifications for adapters. In contrary to Chapter \ref{chap:adaptmt}, we do not investigate language models that we train ourselves, but instead, we focus mainly on the BERT language model. We use the same hyperparameters to Chapter \ref{chap:03} which we describe on \cref{tab:hyp_invest}.

The model and hyperparameters that we use throughout the experiment remain the same as described on \cref{chap:adaptmt}. We use transformer model with seq2seq architecture, and we follow BERT-based configuration to initialize both the encoder and the decoder.

\subsection{Dataset}
As mentioned in the previous section, we focus mainly on machine translation tasks. We have already described on cref{chap:03} that WMT is mainly used for language model training and mixed training for models that were trained from scratch. For machine translation, we focus solely on IWSLT to fine-tune and evaluate our models.

\section{Original BERT}
\subsection{Size of Adapters}
\subsubsection{Experiment Setup and Motivation}
\paragraph{}
In these experiments, we keep the weights of the transformer model intact and only modify the reduction ratio parameter in the adapters. Adapters serve as bottleneck layers that reduce the input size dimension before scaling it back. The reduction ratio is defined as the number of dimensions that we reduce within the bottleneck layer. To be more precise, if we use ``16'' as the reduction ratio, we reduce the original layers by ``16'' and then scale it back to the original size.

We are using various sizes of reduction ratios to compare the result. This reduction aims to see whether we can further benefit from enlarging the adapters' bottleneck size in terms of performance. We use 16, 8, 4, 2, and 1 as the ratio values for this experiment. We compare the results with the baseline BERT that we fine-tuned by only modifying the cross-attention layer. We will refer to this baseline as baseline BERT for the entirety of this work.

\subsubsection{Experiment Results}
In this section, we compare the results of the baseline BERT (only cross-attention fine-tuned) with fine-tuned models with adapters in different reduction ratios. We only fine-tuned the cross-attention layers for the baseline BERT model and kept the rest of the weights intact. We can see in \cref{img:adapt_bert_ratio} that even the smallest model (\texttt{adapt\_bert\_reduc\_16}) can already outperform the baseline by around 2 BLEU. This shows that the adapters can help improve the model's performance by adding only a small number of weights during the fine-tuning.

\begin{figure}[]
    {\includegraphics[width=0.95\textwidth]{img/adapter_bert_baseline_adapters.png}}
    \centering
    \caption{Comparison between baseline BERT model and adapters model with different ratio (16, 8, 4, 2, 1).}
    \label{img:adapt_bert_ratio}
\end{figure}

Furthermore, the difference between the ratios is minimal. It suggests that there is not much benefit in expanding the size of adapters for the normal size BERT. It is possible that it is no longer trivial to append adapters to fine-tune the model for large-sized models such as the original BERT. Further changes may be required to handle the different nature of BERT's output as it is naturally different from the common auto-regressive machine translation objective.

\subsection{Position of Adapters (Encoder vs Decoder)}
\label{sec:posada}
\subsubsection{Experiment Setup and Motivation}
We would like to see the importance of adapters when put in a different place. Since we are working with seq2seq architecture in this work, we would like to see whether only incorporating adapters on one side can already be beneficial and reduce the weight added to the model.

\subsubsection{Experiment Result}
\begin{figure}[]
    {\includegraphics[width=0.95\textwidth]{img/bert_pos.png}}
    \centering
    \caption[Results of ablation study for adapters in the encoder or the decoder.]{Comparison between baseline BERT model and adapters model where the adapters are placed in three different setups: 1) Adapters in both encoder and decoder (\texttt{adapt\_bert\_reduc\_16}); 2) Adapters only in encoder (\texttt{adapter\_bert\_bert}); 3) Adapters only in decoder (\texttt{bert\_adapter\_bert}).}
    \label{img:adapt_bert_pos}
\end{figure}
We see in \cref{img:adapt_bert_pos} that adding adapters just in the encoder part brings an improvement and outperforms the baseline. Adapters only the encoder train the fastest at the beginning, and their final performance is almost the same as if we added the adapters on both sides. For the decoder, on the other hand, we can see that (aside from a more promising start, there is no benefit as there is no improvement in terms of late BLEU scores compared to the baseline. With this finding, we can reduce the cost of fine-tuning further by half when we do not include the adapters on the decoder.
% Adding the adapters on the encoder and fine-tuning it is more cost-effective than the decoder. With this finding, we can reduce the cost of fine-tuning by half when we do not include the adapters on the decoder.

\subsection{True BERT in Encoder vs Decoder}
\label{sec:pospre}
\subsubsection{Experiment Setup and Motivation}
This section investigates the importance of having BERT the pre-training models as the initial weights in either the encoder or decoder. In addition to that, we also expand the experiment further by understanding the correlation of adding adapters on top of the randomly set weights.

We start by defining the definition of the setup in this experiment. The previous chapter introduced an experiment where we instantiate the base Transformer model with only random weights. We then fine-tune the base Transformer model by only updating the adapters and cross-attention layer. In this setup, we are doing experiments in a similar concept. We use random (fixed) weights instead of the original pre-trained BERT weights. We thus have a seq2seq model with the random encoder followed by the BERT decoder and vice versa. At fine-tuning, we update both the adapters in the encoder and decoder and cross-attention layer (or cross-attention layer only in our baseline models).

The purpose of the experiments are:
\begin{itemize}
    \item We want to understand further the importance of the pre-training model when fine-tuned with adapters. By initializing the model with BERT only in one component, we can see whether it is necessary to use BERT on both components when adapters are incorporated.
    \item We want to understand the capability of adapters when either one of the components does not contain useful information (relative to BERT). We would like to see whether the adapters can recover or even outperform some of the performance that we have already gathered from the previous chapters and sections.
\end{itemize}

\subsubsection{Experiment Results}
This section compares models that use adapters in either or both the encoder and decoder while only initializing one of these components with true BERT weights and the other one with (fixed) random weights. The purpose of the experiments is to understand the behaviour of adapters when faced with relatively poor representation in one of the components.

\paragraph{Randomly Set Weights on Encoder}
In this part of the section, we want to answer the main question: "To what extent can the adapters restore the missing gap when the encoder does not contain useful information (relative to BERT)?"

\begin{figure}[h]
    {\includegraphics[width=0.95\textwidth]{img/adapter_bert_randenc.png}}
    \centering
    \caption[Comparison for model with adapters in the decoder and the encoder is initalized with random weights.]{Comparison between baseline BERT model and adapters model where the adapters are placed in three different setups: 1) Adapters in both encoder and decoder (\texttt{adapt\_bert\_reduc\_16}); 2) Adapters only in encoder (\texttt{adapter\_bert\_bert}); 3) Adapters only in decoder (\texttt{bert\_adapter\_bert}) and the decoder is initialized with BERT while the encoder is initialized with random numbers.}
    \label{img:adapt_bert_randenc}
\end{figure}

We can see from \cref{img:adapt_bert_randenc} that when adapters are appended in both components, we get to almost 20 in the BLEU score. This is relatively higher than the other two setups (adapters only in the encoder and only in the decoder). However, compared to the baseline, we are missing 4 points in BLEU when we set the weights on the encoder as entirely random. This means that the base encoder model was missing relatively essential information that the adapters can not simply restore during the fine-tuning.

We further focus on the adapters' performance compared to the baseline model that is only fine-tuned on the cross-attention layer to see whether there is any benefit in using adapters or simply fine-tuning the cross-attention already enough. We can see that the model that was only fine-tuning the cross-attention layer can not learn at all, while the adapters can perform significantly better. This marks the capability of the adapter when faced with a randomly set encoder.

When the adapters are removed from the decoder, we see a degradation in performance in about 1 BLEU. However, when the adapters are removed from the encoder, we can see the performance is completely depleted during the training. We also see the same behaviour in the next section when the weights on the decoder are set randomly. This tells us that there is some incompatibility in the weights (random and BERT) where it is not trivial to fine-tune the cross-attention layer without further adjustment to the base model's weights.

\paragraph{Randomly Set Weights on Decoder}
Similar to the previous section, the main question in this experiment is, "To what extent can the adapters restore the missing gap when the decoder does not contain useful information (relative to BERT)?"

In contrast to when the randomly set weights are on the encoder side, we can see from \cref{img:adapt_bert_randdec} that the model has comparable performance to the one we have on the BERT baseline. This tells us that the pre-training weights in the encoder are more critical than in the decoder when we have adapters on both sides. However, when removing the adapters on the encoder side, we see similar performance as in the previous section, where the performance drops to zero in the middle of training. This furthers our arguments that adapters are necessary to adjust the weights in the model so that the cross-attention layer can work properly.

\begin{figure}[h]
    {\includegraphics[width=0.95\textwidth]{img/adapter_bert_randdec.png}}
    \centering
    \caption[Comparison for model with adapters in the decoder and the decoder is initalized with random weights]{Comparison between baseline BERT model and adapters model where the adapters are placed in three different setups: 1) Adapters in both encoder and decoder (\texttt{adapt\_bert\_reduc\_16}); 2) Adapters only in encoder (\texttt{adapter\_bert\_bert}); 3) Adapters only in decoder (\texttt{bert\_adapter\_bert}) and the encoder is initialized with BERT while the decoder is initialized with random numbers.}
    \label{img:adapt_bert_randdec}
\end{figure}

On the other hand, when we remove the adapters from the decoder side, we can see that the performance is not as bad as when the adapters are removed from the encoder, but we still see a reduction in performance. We see around $<$ 1 BLEU when the model reaches 400k steps in the training stage.

\section{BERT Size Reduction}
\subsection{Zeroing Columns}
\subsubsection{Experiment Setup and Motivation}
In this experiment, we will focus on the soft reduction of BERT weights by zeroing the weights on every even index column and row in both the transformer body as well as in the embedding weights. We load pre-trained BERT weights, manually edit them and then continue the experiments by fine-tuning the cross-attention and adapters. We further refer to this setup as \texttt{zbert} for the rest of this writing.

Besides removing the columns, we also perform experiments where we put the adapters either on the encoder or the decoder. The goal of this particular experiment is to understand the behaviour of the model when the pre-trained models are replaced with this particular setup.

\subsubsection{Comparison with BERT Baseline (Full BERT Fine-tuning)}
We first compare the \texttt{zbert} model without adapters and only fine-tuning the cross-attention layers. We use \texttt{zbert} weights on both encoder and decoder so that it is comparable to the model that uses full-weight BERT. We use the full-weight BERT as the baseline in this experiment.

\begin{figure}[]
    {\includegraphics[width=0.95\textwidth]{img/baseline_zbert.png}}
    \centering
    \caption{Comparison between baseline BERT model and baseline \texttt{zbert} models.}
    \label{img:baseline_zbert}
\end{figure}1

\begin{figure}[]
    {\includegraphics[width=0.95\textwidth]{img/adapter_zbert.png}}
    \centering
    \caption{Comparison between baseline BERT model, baseline \texttt{zbert} and adapters \texttt{zbert} models.}
    \label{img:adapter_zbert}
\end{figure}

We can see in \cref{img:baseline_zbert} that we are losing performance of about 4 BLEU. This is significant as we are losing various essential features from the original BERT model. To see whether we can recover some of the performance with adapters, we continue our experiment by fine-tuning \texttt{zbert} model that is instantiated on both encoder and decoder sides with adapters. We can see from \cref{img:adapter_zbert} that we only managed to recover 1 BLEU with a reduction ratio of 16.

\begin{figure}[]
    {\includegraphics[width=0.95\textwidth]{img/adapter_zbert_ratio.png}}
    \centering
    \caption{Comparison between baseline BERT model and different reduction ratio of \texttt{zbert} models.}
    \label{img:adapter_zbert_ratio}
\end{figure}

From \cref{img:adapter_zbert_ratio}, when we increase the size of the reduction ratio, initially, we can see some improvement in the BLEU score compared to the higher ratio. However, they eventually converge to a similar performance by the end of training with no significant difference between different ratios. From this result, we can understand that depending on the base pre-trained model, adapters still have a limitation in achieving certain performance.

\paragraph{Adapters Position}
In this section, we are trying to understand whether the position of both adapters and the pre-training models affect the model performance, similar to what we have seen in \cref{sec:posada}. We use a similar setup as in the previous section, with the exception that we use \texttt{zbert} as the pre-trained model instead of the original BERT model.

\begin{figure}[h]
    {\includegraphics[width=0.95\textwidth]{img/zbert_pos.png}}
    \centering
    \caption[Comparison between baseline BERT and \texttt{zbert} models.]{Comparison between baseline BERT model, baseline \texttt{zbert} model, adapters in both encoder and decoder of \texttt{zbert} model (\texttt{adapt\_zbert\_reduc\_16}), adapters only in encoder of \texttt{zbert} model (\texttt{adapter\_zbert\_zbert}), and adapters only in decoder of \texttt{zbert} model (\texttt{zbert\_adapter\_zbert}).}
    \label{img:zbert_pos}
\end{figure}

We can see from \cref{img:zbert_pos} that when we include adapters on both encoder and decoder, we can outperform the baseline \texttt{zbert} in around 2 BLEU points. This shows that similar to the models that use BERT as the pre-trained models, the adapters can help to improve the performance further even though some of the information is already missing in the base model.

Furthermore, we can also see that similar to the BERT model that was fine-tuned with adapters, using adapters only on the encoder side performs much better than on the decoder side. Other than that, we can also see that incorporating adapters only on the encoder side helps the model achieve better performance faster than the model that uses adapters on both sides. This further support our hypothesis that updating the representation on the encoder side is more beneficial. Looking deeper at the model with adapters only on the decoder, the performance is close to the baseline model, where we only fine-tune the cross-attention layer. This could mean that fine-tuning the decoder may not be enough to achieve better performance when the representation from the source side is constant.

\subsection{Model Down-scaling}
\subsubsection{Experiment Setup and Motivation}
This experiment is the follow-up from \texttt{zbert} where we zeroed out half of the elements in the matrices. To be more specific, we are now completely removing those elements from the matrix instead of just zeroing out the elements. The way we do this is similar to the one we do on \texttt{zsbert}. We remove the matrix elements on every even column and row in both the transformer body as well as the embedding weights. We similarly do the weights processing offline before using it as the pre-training base model. For the rest of this writing, we refer to this setup as \texttt{zsbert}.

Furthermore, we also follow a similar setup as in \cref{sec:posada} where we experiment with the position of the adapters. The goal of this experiment is that we would like to understand the behaviour of the model compared to the baseline as well as \texttt{zbert}.

\subsubsection{Comparison with BERT Baseline and \texttt{zbert}}

\begin{figure}[h]
    {\includegraphics[width=0.95\textwidth]{img/baseline_zsbert.png}}
    \centering
    \caption{Comparison between baseline BERT model and baseline \texttt{zsbert} model.}
    \label{img:baseline_zsbert}
\end{figure}

\begin{figure}[]
    {\includegraphics[width=0.95\textwidth]{img/adapter_zsbert.png}}
    \centering
    \caption{Comparison between baseline BERT model, baseline \texttt{zbert}, baseline \texttt{zsbert} and adapters \texttt{zsbert} models.}
    \label{img:adapter_zsbert}
\end{figure}


We begin with the comparison of \texttt{zsbert} with the BERT baseline. We see in \cref{img:baseline_zsbert} that the performance degrades by almost 10 BLEU points. This is also significantly worse than \texttt{zbert} where we only lose 5 BLEU. After some investigation, we realized that removing weights from the network is not straightforward as the computation of layer normalization depends on the matrix size and the adapter scaling factor will also be different. With manual evaluation, we found a slight difference between the only zeroed weights and the completely removed weights. This slight difference in the output of layer normalization gets propagated to the top layers, causing the result to differ significantly.

% Next, we study the interplay of model down-scaling and adapters. We can see in \cref{img:adapter_zsbert} that adapters with a 16 ratio manage to improve the performance up to 6 BLEU compared to \texttt{zbert} without adapters.
% We also notice a leap in final performance when we compare the adapter model with an equal reduction ratio (16) between \texttt{zbert} and \texttt{zsbert}. We can see that initially \texttt{zsbert} performs worse than \texttt{zbert}. After some steps, we can see the performance in \texttt{zbert} starting to stall but not in \texttt{zsbert}. We hypothesize that this relates to a similar reason that we stated in the original BERT model, where we could not see any improvement when increasing the reduction ratio. It is possible that when we reduce the size of the original pre-trained model, the adapters manage to adjust the flow of information within the network and better replace the missing information with new knowledge that is more important for solving the task.

\begin{figure}[]
    {\includegraphics[width=0.95\textwidth]{img/adapter_zsbert_ratio.png}}
    \centering
    \caption{Comparison between baseline BERT model and different reduction ratio of \texttt{zsbert} models.}
    \label{img:adapter_zsbert_ratio}
\end{figure}

In \cref{img:adapter_zsbert_ratio}, when the adapters reduction ratio is further reduced, we can see that the performance is also improving up to the point where it has close performance to the baseline model. This remarks a prominent result as we can see from \cref{tab:numvars} that the total number of weights (including adapters) is reduced significantly.

\begin{table*}[]
    \centering
    \begin{tabular}{@{}cccc@{}}
        \toprule
        \textbf{Name}                                                               &
        \textbf{\begin{tabular}[c]{@{}c@{}}\# Trainable\\ Variables\end{tabular}}   &
        \textbf{\begin{tabular}[c]{@{}c@{}}\# Untrainable\\ Variables\end{tabular}} &
        \textbf{\begin{tabular}[c]{@{}c@{}}\# Total\\ Variables\end{tabular}}                                                \\ \midrule
        \textbf{Adapters ratio 16}                                                  & 7.736.826  & 95.143.296  & 102.880.122 \\
        \textbf{Adapters ratio 8}                                                   & 8.179.770  & 95.143.296  & 103.323.066 \\
        \textbf{Adapters ratio 4}                                                   & 9.065.658  & 95.143.296  & 104.208.954 \\
        \textbf{Adapters ratio 2}                                                   & 10.837.434 & 95.143.296  & 105.980.730 \\
        \textbf{Adapters ratio 1}                                                   & 14.380.986 & 95.143.296  & 109.524.282 \\
        \textbf{Normal BERT}                                                        & 28.990.078 & 218.819.328 & 247.809.406 \\ \bottomrule
    \end{tabular}
    \caption{Total trainable variables in \texttt{zsbert} with adapters on different ratio vs normal BERT model}
    \label{tab:numvars}
\end{table*}

\subsubsection{Adapters Position}

\begin{figure}[h]
    {\includegraphics[width=0.95\textwidth]{img/zsbert_pos.png}}
    \centering
    \caption[Comparison between baseline BERT and \texttt{zsbert} models.]{Comparison between baseline BERT model, baseline \texttt{zsbert} model, adapters in both encoder and decoder of \texttt{zsbert} model (\texttt{adapt\_zsbert\_reduc\_16}), adapters only in encoder of \texttt{zsbert} model (\texttt{adapter\_zsbert\_zsbert}), and adapters only in decoder of \texttt{zsbert} model (\texttt{zsbert\_adapter\_zsbert}).}
    \label{img:zsbert_pos}
\end{figure}

From \cref{img:zsbert_pos}, similar to the \texttt{zbert} experiments, we can see similar behaviour as models that are fine-tuned with adapters outperform the baseline \texttt{zsbert} and \texttt{zbert} models. However, compared to \texttt{zbert} experiments, we notice a bigger improvement in \texttt{zsbert}'s final performance. In \texttt{zbert}, the difference between baseline and adapters is in the range of 5 BLEU. On the other hand, in \texttt{zsbert} we see the improvement is in the range of 8 BLEU. This result is particularly interesting for us as we recall from the baseline experiments that due to the numerical error from the layer normalization, we expect the difference to be similar to \texttt{zbert} and have lower performance than we currently have.

We deep-dive further in \cref{img:zbert_vs_zsbert} to show the comparison between adapters in \texttt{zbert} and \texttt{zsbert}. We compare the adapters model with an equal reduction ratio (16) between these two setups. We notice a leap in the final performance when we compare the adapter model with an equal reduction ratio (16) between \texttt{zbert} and \texttt{zsbert}. We can see that initially \texttt{zsbert} performs worse than \texttt{zbert}. After some steps, we can see the performance in \texttt{zbert} starting to stall but not in \texttt{zsbert}. We hypothesize that this relates to a similar reason that we stated in the original BERT model, where we could not see any improvement when increasing the reduction ratio. It is possible that when we reduce the size of the original pre-trained model, the adapters manage to adjust the flow of information within the network and better replace the missing information with new knowledge that is more important for solving the task.

\begin{figure}[h]
    {\includegraphics[width=0.95\textwidth]{img/zbert_vs_zsbert.png}}
    \centering
    \caption{Comparison adapters performance in \texttt{zsbert} and \texttt{zbert}. Both are using reduction ratio 16 and the adapters are placed on encoder and decoder.}
    \label{img:zbert_vs_zsbert}
\end{figure}

We see similar behaviour as we see on \texttt{zbert} experiments in regards to the position of the adapters. In \cref{img:zsbert_pos}, the benefit of incorporating adapters on the encoder side is also apparent and outperforms the decoder counterpart. We can also see a similar behaviour where eventually, the model's performance with adapters on the encoder outperforms the model with adapters on both sides. Furthermore, we also see similar behaviour as in \texttt{zbert} for models with adapters where the performance is very close to the baseline and not improving as much as the encoder side. We hypothesize the same reason as we have stated in \texttt{zbert} could apply in \texttt{zsbert} as well. Essentially, we need to modify the representation on the encoder side in order to achieve better performance.


\chapter*{Conclusion}
\addcontentsline{toc}{chapter}{Conclusion}
This thesis has explored various ways to utilize BERT with adapters in machine translation. We start by understanding the impact of pre-training data in different domains and the contribution of different volumes in the pre-training data. We continue the study by leveraging different techniques to understand the impact of the pre-trained representations by shuffling the original BERT weights and using randomly set weights. We then conduct the experiments to understand the importance of adapters in either the encoder or the decoder by showing the performance of adapters when they are removed from either of the components. Finally, we perform reduction experiments where we reduce the size of BERT by manually removing the weights either by zeroing out some of the values in the matrices or completely deleting them.

Experiments in Chapter \ref{chap:adaptmt} show that with the proper initialization, adapters can help achieve better performance than training the models from scratch while training far fewer weights than the original model. We further show that even with random fixed weights in the main part of the model, the adapters and cross-attention can recover and achieve performance similar to one of the baseline models.

In the subsequent experiments in Chapter \ref{chap:adaptefct}, we find that fine-tuning adapters on the encoder side is more important than in the decoder. We also see a similar behaviour when we use the original BERT weights only on the encoder or the decoder and fixed random weights on the other part. Interestingly, when the adapters were injected only to the decoder, with the encoder pre-trained or random, the performance dropped to zero. In other words, a fitting encoder is critical.

We further studied the behaviour of adapters when we tried to down-scale the pre-trained model size. In our experiments, we found that the model with just half of the weights, such as our \texttt{zsbert}, can closely match the performance of the baseline model, the model that uses BERT in both of the encoder and the decoder and only fine-tuning the cross-attention and output layers. Finally, we also observe that we can increase further the effectiveness of adapters in \texttt{zsbert} by only incorporating adapters on the encoder side.

We see two practical applications from our findings:
\begin{itemize}
    \item Initializing just the encoder with the pre-trained weights such as BERT (with a fixed random decoder) and fine-tuning with adapters could be helpful when targeting low-resource languages. The random decoder is created trivially and no large target side monolingual corpus is needed.
    \item Reducing the pre-trained BERT to half its size and fine-tuning with adapt\-ers provide useful GPU memory savings while keeping a similar performance as the baseline model.
\end{itemize}

In summary, our experiments show the potential of adapters in machine translation setup. We understand from the experiments that fine-tuning adapters with randomly set weights in the base pre-trained network can achieve similar results as training the entire transformer model with BERT configuration. Furthermore, the down-scaling experiments also show that with a random weight reduction technique, we can reduce the size of BERT and achieve similar performance as the BERT model that was fine-tuned by only modifying the cross-attention layer.

%%% Bibliography
\include{bibliography}

%%% Figures used in the thesis (consider if this is needed)
\listoffigures

%%% Tables used in the thesis (consider if this is needed)
%%% In mathematical theses, it could be better to move the list of tables to the beginning of the thesis.
\listoftables

%%% Abbreviations used in the thesis, if any, including their explanation
%%% In mathematical theses, it could be better to move the list of abbreviations to the beginning of the thesis.
% \chapwithtoc{List of Abbreviations}

%%% Attachments to the master thesis, if any. Each attachment must be
%%% referred to at least once from the text of the thesis. Attachments
%%% are numbered.
%%%
%%% The printed version should preferably contain attachments, which can be
%%% read (additional tables and charts, supplementary text, examples of
%%% program output, etc.). The electronic version is more suited for attachments
%%% which will likely be used in an electronic form rather than read (program
%%% source code, data files, interactive charts, etc.). Electronic attachments
%%% should be uploaded to SIS and optionally also included in the thesis on a~CD/DVD.
%%% Allowed file formats are specified in provision of the rector no. 72/2017.
\appendix
% \chapter{Attachments}

% \section{First Attachment}

\openright
\end{document}
